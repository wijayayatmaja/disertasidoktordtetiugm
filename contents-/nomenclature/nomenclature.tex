\setlength{\LTleft}{0pt}
{\setstretch{0.9667}
\begin{longtable}{llp{288pt}}
$A$								& = & himpunan yang memuat skenario-skenario pendistribusian PLTS atap\\
$A^m$							& = & fungsi keluaran untuk skenario pendistribusian PLTS atap ke-$m$\\
$a^m_k$							& = & fungsi keluaran untuk proses pendistribusian PLTS atap ke-$k$ dari skenario pendistribusian PLTS atap ke-$m$\\
$u$								& = & jumlah skenario pendistribusian PLTS atap\\
$w$								& = & jumlah proses pendistribusian PLTS atap untuk setiap skenario pendistribusian PLTS atap\\
$C^m$							& = & himpunan yang memuat pelanggan-pelanggan dengan PLTS atap untuk skenario pendistribusian PLTS atap ke-$m$\\
$c^m_k$							& = & pelanggan dengan PLTS atap untuk skenario pendistribusian PLTS atap ke-$m$ dan proses pendistribusian PLTS atap ke-$k$\\
$X^m$							& = & himpunan yang memuat ukuran-ukuran daya PLTS atap untuk skenario pendistribusian PLTS atap ke-$m$\\
$x^m_k$							& = & ukuran daya PLTS atap untuk skenario pendistribusian PLTS atap ke-$m$ dan proses pendistribusian PLTS atap ke-$k$\\
$D$								& = & himpunan yang memuat data historis berupa ukuran-ukuran daya PLTS atap yang diintegrasikan ke jaringan distribusi\\
$L^m$							& = & himpunan yang memuat lokasi-lokasi integrasi PLTS atap\\
$l^m_k$							& = & lokasi integrasi PLTS atap untuk skenario pendistribusian PLTS atap ke-$m$ dan proses pendistribusian PLTS atap ke-$k$\\
$CP^m$							& = & himpunan yang memuat level-level penetrasi pelanggan\\
$cp^m_k$						& = & level penetrasi pelanggan untuk skenario pendistribusian PLTS atap ke-$m$ dan proses pendistribusian PLTS atap ke-$k$\\
$XP^m$							& = & himpunan yang memuat level-level penetrasi PLTS atap\\
$xp^m_k$						& = & level penetrasi PLTS atap untuk skenario pendistribusian PLTS atap ke-$m$ dan proses pendistribusian PLTS atap ke-$k$\\
$h^m$							& = & kapasitas \textit{hosting} PLTS atap untuk skenario pendistribusian PLTS ke-$m$\\
$Pr$	                        & = & fungsi probabilitas \\
$Pr[T^m=\mathit{t^m_k}]$ 		& = & probabilitas terjadinya pelanggaran pada proses pendistribusian PLTS atap ke-$k$ dan skenario pendistribusian PLTS atap ke-$m$\\
$h_{min}$						& = & kapasitas \textit{hosting} PLTS atap minimum dari semua skenario pendistribusian PLTS atap\\
$P^{norm}_{PV}$                 & = & daya keluaran PLTS setelah dinormalisasi terhadap nilai puncaknya\\
$Acc^{\epsilon}$                & = & ukuran akurasi pada parameter tolerasi sebesar $\epsilon$\\
$P_{source}$ 					& = & daya aktif operasi penyulang\\
$Q_{source}$ 					& = & daya reaktif operasi penyulang\\
$P_{loss_d}$ 					& = & rugi-rugi daya aktif ke-$d$\\
$Q_{loss_d}$ 					& = & rugi-rugi daya reaktif ke-$d$\\
$\mathit{v_b}$                  & = & tegangan pada \textit{bus} ke-$b$ \\
$\mathit{PF_{feeder}}$          & = & faktor daya operasi penyulang \\
$pf_i$							& = & faktor daya PLTS atap ke-$i$\\
PV                              & = & \textit{photovoltaic}\\
PLTS                            & = & pembangkit listrik tenaga surya\\
MNP								& = & \textit{multistage non-finite population}\\
RV                              & = & \textit{risk of violation}\\
SV                              & = & \textit{system without violation}
\end{longtable}
}