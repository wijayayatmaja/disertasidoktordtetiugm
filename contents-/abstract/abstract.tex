Since large-scale rooftop photovoltaic (PV) penetration could cause performance degradation of the distribution grid, a rooftop photovoltaic hosting capacity assessment needs to be studied. For this reason, the objective of this research is to develop a hosting capacity assessment method based on Monte Carlo to increase the accuracy and reduce the negative impact of the penetration. To increase the accuracy, a development of the interval time-series analysis framework is provided. Furthermore, a power factor variation of rooftop PV systems is considered to reduce the negative impact. In assessing hosting capacity, the proposed method is performed on the basis of reverse power flow, overvoltage, and feeder operating power factor limit. To consider actual issues in a distribution grid, an actual feeder from Yogyakarta province of Indonesia is employed. The results show that the proposed method could increase the accuracy and reduce the negative impact of the penetration for the same penetration level. Moreover, rooftop photovoltaic hosting capacity of 13.49\% of full load is obtained. In addition, the feeder operating power factor is observed as the most restrictive limit.

\noindent\textbf{Keywords---}Accuracy, negative impact, distribution grid, Monte Carlo, hosting capacity, rooftop PV penetration.