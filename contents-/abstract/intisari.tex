{\setstretch{1}
Pada level tertentu, penetrasi pembangkit listrik tenaga surya (PLTS) tersebar dapat memberikan dampak negatif pada jaringan distribusi tegangan rendah, seperti pelanggaran tegangan lebih, aliran daya balik, dan pelanggaran kapasitas termal konduktor. Sementara itu, dari tahun ke tahun, penetrasi PLTS tersebar tercatat semakin tinggi. Sebagai konsekuensinya, penentuan kapasitas \textit{hosting} stokastik perlu dilakukan, khususnya oleh perencana sistem distribusi dalam menghadapi kenaikan penetrasi PLTS. Agar perencana sistem distribusi tersebut dapat memutuskan langkah strategis yang tepat, dibutuhkan metode penentuan kapasitas \textit{hosting} yang akurat dengan model yang representatif dengan kondisi riil penetrasi. Oleh karena itu, penelitian disertasi ini bertujuan mengembangkan metode penentuan kapasitas \textit{hosting} stokastik berbasis pendekatan Markov Chain Monte Carlo untuk meningkatkan akurasi dengan model yang representatif terhadap kondisi riil penetrasi PLTS. Dalam hal ini, kapasitas \textit{hosting} stokastik dihitung dengan memperhatikan pelanggaran terhadap batas tegangan lebih. Untuk meningkatkan akurasi metode, penelitian ini mengembangkan pendekatan Markov Chain Monte Carlo dengan \textit{layered hidden Markov model}. Karakteristik solusi dari \textit{layered hidden Markov model} yang mempunyai probabilitas-probabilitas transisi/keluaran digunakan untuk pergerakan dari suatu \textit{state} ke \textit{state} berikutnya. Terkait hal ini, ketidakpastian ukuran daya dan lokasi telah dikembangkan pada penelitian sebelumnya. Sebagai pengembangannya, penelitian disertasi ini menggunakan pertimbangan kategori pelanggan, kategori penetrasi PLTS dan PLTS-baterai, serta kategori daya PLTS dan PLTS-baterai. Untuk mengukur kinerja metode yang diusulkan, sebuah studi kasus dilakukan pada penyulang IEEE 8500-node. Studi perbandingan diterapkan dengan membandingkan pendekatan Markov Chain Monte Carlo dan Monte Carlo. Mean absolute error (MAE) diterapkan untuk mengukur akurasi dari metode yang diusulkan pada studi perbandingan tersebut. Hipotesis dari penelitian ini adalah metode yang diusulkan terbukti dapat meningkatkan akurasi, dibandingkan dengan metode sebelumnya yaitu Monte Carlo.

\noindent\textbf{Kata kunci---}Markov Chain Monte Carlo, kapasitas \textit{hosting} \vspace*{-5.5pt} stokastik, penetrasi PLTS dan PLTS-baterai, akurasi.
}