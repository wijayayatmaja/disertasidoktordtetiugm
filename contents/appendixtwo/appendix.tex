\noindent Matriks berikut:
\begin{equation*}
	\left(
	\begin{array}{cccc}
	a+b+c & uv & x-y & 27\\
	a+b & u+v & z & 134
	\end{array}\right)
\end{equation*}

\noindent ditulis dalam kode program sebagai:
\begin{verbatim}
	\begin{equation*}
	\left(
	\begin{array}{cccc}
	a+b+c & uv & x-y & 27\\
	a+b & u+v & z & 134
	\end{array} \right)
	\end{equation*}
\end{verbatim}

\noindent Matriks berikut:
\begin{equation*}
\begin{matrix}  0 &  1 \\ 
1 &  0 \end{matrix}
\end{equation*}

\noindent ditulis dalam kode program sebagai:
\begin{verbatim}
\begin{equation*}
\begin{matrix}  0 &  1 \\ 
1 &  0 \end{matrix}
\end{equation*}
\end{verbatim}

\noindent Matriks berikut:
\begin{equation*}
\begin{pmatrix} 0 & -i \\
 i &  0 \end{pmatrix}
\end{equation*}

\noindent ditulis dalam kode program sebagai:
\begin{verbatim}
\begin{equation*}
\begin{pmatrix} 0 & -i \\
 i &  0 \end{pmatrix}
\end{equation*}
\end{verbatim}

\noindent Matriks berikut:
\begin{equation*}
\begin{bmatrix} 0 & -1 \\ 
1 &  0 \end{bmatrix}
\end{equation*}

\noindent ditulis dalam kode program sebagai:
\begin{verbatim}
\begin{equation*}
\begin{bmatrix} 0 & -1 \\ 
1 &  0 \end{bmatrix}
\end{equation*}
\end{verbatim}

\noindent Matriks berikut:
\begin{equation*}
\begin{Bmatrix} 1 &  0 \\ 
0 & -1 \end{Bmatrix}
\end{equation*}

\noindent ditulis dalam kode program sebagai:
\begin{verbatim}
\begin{equation*}
\begin{Bmatrix} 1 &  0 \\ 
0 & -1 \end{Bmatrix}
\end{equation*}\end{verbatim}

\noindent Matriks berikut:
\begin{equation*}
\begin{vmatrix} a &  b \\ 
c &  d \end{vmatrix}
\end{equation*}

\noindent ditulis dalam kode program sebagai:
\begin{verbatim}
\begin{equation*}
\begin{vmatrix} a &  b \\ 
c &  d \end{vmatrix}
\end{equation*}\end{verbatim}

\noindent Matriks berikut:
\begin{equation*}
\begin{Vmatrix} i &  0 \\ 
0 & -i \end{Vmatrix}
\end{equation*}

\noindent ditulis dalam kode program sebagai:
\begin{verbatim}
\begin{equation*}
\begin{Vmatrix} i &  0 \\ 
0 & -i \end{Vmatrix}
\end{equation*}\end{verbatim}
