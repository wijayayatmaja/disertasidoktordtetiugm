%!TEX root = /Users/wyatmaja/Documents/Doctor/proposalhc/dissertationproposal.tex
\chapter{KEMAJUAN PENELITIAN}
Bab ini menyediakan kemajuan-kemajuan penelitian yang sudah didapatkan. Penelitian ini dimulai dengan merancang metode penentuan kapasitas \textit{hosting} stokastik untuk penetrasi PLTS tersebar dengan menggunakan Markov Chain Monte Carlo sederhana, belum menggunakan \textit{hidden Markov model}.

\section{Rancangan Skenario Penetrasi PLTS Tersebar}

%\section{Pengkajian Pustaka dan Persiapan Data Masukan}
%\section{Formulasi Permasalahan, Perancangan Metode, dan Pengerjaan Simulasi}
%\subsection{Kapasitas \textit{Hosting} Stokastik}
%\subsection{Batas Kinerja Operasi}
%\subsection{Kerangka Analisis Runtun Waktu}
%\subsection{Perancangan Metode Penentuan Kapasitas \textit{Hosting} Stokastik}
%\subsubsection{Perancangan Skenario Penetrasi PLTS Tersebar}
\begin{enumerate}
    \item \textit{Skenario penetrasi}:
    Skenario penetrasi $A^m$ adalah skenario ke-$m$ dalam mempenetrasikan PLTS ke jaringan distribusi tenaga listrik. Skenario penetrasi $A^m$ dirumuskan dalam himpunan $A$, yaitu
    \begin{equation}\label{pas1}
    %A=\{A^m\ |\ m\in M\},
    A=\{A^1,A^2,...,A^m,...,A^u\}.
    \end{equation}
    
    Dalam penelitian ini, jumlah semua skenario penetrasi $u$ juga merepresentasikan jumlah simulasi Markov Chain Monte Carlo.
    
    \item \textit{States dari Markov Chain Monte Carlo}:
    Pada penelitian ini, \textit{states} dari Markov Chain Monte Carlo merepresentasikan kluster yang mungkin dari pelanggan, yaitu
    \begin{equation}\label{smc1}
    Y=\{y|y \text{ adalah kluster dari pelanggan}\}.
    \end{equation} 
    
    Dalam hal ini, \textit{state} $y_i$ adalah \textit{state} sebelumnya dari $y_j$. Keduanya merupakan nilai keluaran dari $y$.
    
    \item \textit{Proses penetrasi}:
    Skenario penetrasi ke-$m$ terdiri atas sejumlah PLTS dengan ukuran daya dan lokasi penetrasi yang berbeda-beda. Untuk setiap skenario penetrasi, proses melakukan penetrasi ke-$\{j,k\}$ yang merepresentasikan ukuran daya dan lokasi penetrasi disebut sebagai proses penetrasi $a^m_{j,k}$. Hal ini dirumuskan sebagai
    \begin{equation}\label{pas2}
    %A^m=\{a^m_{j,k}\ |\ m\in M\wedge k\in K\}.
    A^m=\{a^m_{j,1},a^m_{j,2},...,a^m_{j,k},...,a^m_{j,w}\}.
    \end{equation} 
    
    Dengan skenario penetrasi sebanyak $u$ dan proses penetrasi sebanyak $w$ maka dihasilkan total kombinasi acak dari ukuran daya dan lokasi penetrasi sebanyak $u\bigcdot w$.
    
    \item \textit{Pelanggan penetrasi}:
    Dalam penelitian ini, pelanggan penetrasi $c^m_{j,k}$ didefinisikan sebagai pelanggan saat ini yang dipilih secara acak untuk mengintegrasikan PLTS ke jaringan distribusi tenaga listrik. Pelanggan penetrasi dipilih untuk proses penetrasi ke-$\{m,j,k\}$. Hal ini didefinisikan sebagai
    \begin{equation}\label{cpv1}
    %\mathit{C^m_j}=\{c^m_{j,k}\ |\ m\in M\wedge k\in K\wedge j\in \Z^+\}.
    \mathit{C^m_j}=\{c^m_{j,1},c^m_{j,2},...,c^m_{j,k},...,c^m_{j,w}\}.
    \end{equation}
    
    Untuk setiap proses penetrasi ke-${k}$, dipilih satu pelanggan penetrasi $c^m_{j,k}$. Oleh karena itu, jika simulasi Markov Chain Monte Carlo sebanyak $u$ dikerjakan maka terdapat skenario penetrasi sebanyak $u$ untuk setiap pelanggan penetrasi. Hal ini dirumuskan sebagai
    \begin{equation}\label{cpv2}
    \mathit{C^k_j}=\{c^k_{j,1},c^k_{j,2},...,c^k_{j,m},...,c^k_{j,u}\}.
    \end{equation}
    
    \item \textit{Ukuran daya penetrasi}:
    Ukuran daya penetrasi PLTS $x^m_{j,k}$ ditentukan secara bersamaan. Untuk pelanggan yang hanya mengintegrasikan PLTS, ukuran daya yang digunakan hanya PLTS. Pada pelanggan yang mengintegrasikan PLTS dan baterai, baik PLTS maupun baterai ditentukan ukuran daya masing-masing. Ukuran daya penetrasi $x^m_{j,k}$ didasarkan pada skenario ke-$m$, kluster ke-$j$, dan proses penetrasi ke-$k$. Dengan kata lain, ukuran daya penetrasi $x^m_{j,k}$ adalah ukuran daya dari PLTS untuk pelanggan penetrasi $c^m_{j,k}$. Ukuran daya penetrasi dipilih dari data historis $D$ melalui proses Markov Chain Monte Carlo. Ukuran daya penetrasi $x^m_{j,k}$ dirumuskan dalam himpunan ukuran daya penetrasi $X^m_{j}$ sebagai
    %\begin{equation}\label{xpv1}
    %x^m_{j,k}=\{x^m_{j,k}\ \text{is random value}|y_{j-1} < x^m_{j,k} %\leq y_j\},
    %\end{equation}
    \begin{equation}\label{xpv2}
    %\mathit{X^m_j}=\{x^m_{j,k}\ |\ y_{j-1} < x^m_{j,k} \leq y_j\wedge x^m_{j,k}\in D\wedge m\in M\wedge k\in K\wedge y_j\in Y\wedge j\in \Z^+\}.
    \mathit{X^m_j}=\{x^m_{j,1},x^m_{j,2},...,x^m_{j,k},...,x^m_{j,w}\},
    \end{equation}
    dan dapat dirumuskan sebagai
    \begin{equation}\label{xpv3}
    \mathit{X^k_j}=\{x^k_{j,1},x^k_{j,2},...,x^k_{j,m},...,x^k_{j,u}\}.
    \end{equation}
    
    \item \textit{Level penetrasi pelanggan}:
    Untuk setiap skenario penetrasi, level penetrasi pelanggan $cp^m_k$ adalah jumlah total dari pelanggan-pelanggan penetrasi dari awal simulasi sampai proses penetrasi ke-$k$. Secara matematis, level penetrasi pelanggan $cp^m_k$ didefinisikan sebagai
    \begin{equation}\label{cpl1}
    cp^m_k=k,
    \end{equation}
    dan untuk semua penetrasi,
    \begin{equation}\label{cpl2}
    cp^m_w=w,
    \end{equation}
    kemudian, himpunan level penetrasi pelanggan $\mathit{CP^m}$ dirumuskan sebagai
    \begin{equation}\label{cpl3}
    %\mathit{CP^m}=\{cp^m_k\ |\ m\in M\wedge k\in K\}.
    \mathit{CP^m}=\{cp^m_1,cp^m_2,...,cp^m_k,...,cp^m_w\}.
    \end{equation}
    \item \textit{Level penetrasi daya}:
    Jumlah total ukuran daya yang telah dipenetrasikan ke jaringan distribusi tenaga listrik ke-$\{m,k\}$ didefinisikan sebagai level penetrasi daya $xp^m_k$. Level penetrasi daya $xp^m_k$ didefinisikan sebagai
    \begin{equation}\label{xpl1}
    xp^m_k=\sum_{k=1}^k x^m_{j,k},
    \end{equation}
    jika telah tercapai 100\% penetrasi maka
    \begin{equation}\label{xpl2}
    \mathit{xp^m_w}=\sum_{k=1}^w x^m_{j,k},
    \end{equation}
    kemudian, himpunan level penetrasi daya $\mathit{XP^m}$ dirumuskan sebagai
    \begin{equation}\label{xpl3}
    %\mathit{XP^m}=\{xp^m_k\ |\ m\in M\wedge k\in K\}.
    \mathit{XP^m}=\{xp^m_1,xp^m_2,...,xp^m_k,...,xp^m_w\}.
    \end{equation}
    
    \item \textit{Lokasi penetrasi}:
    Lokasi penetrasi $l^m_{j,k}$  didasarkan pada lokasi pelanggan penetrasi $c^m_{j,k}$ yang dipilih. Oleh karena itu, lokasi penetrasi $l^m_{j,k}$ dirumuskan sebagai
    \begin{equation}\label{lpv1}
    l^m_{j,k}=location(c^m_{j,k}),
    \end{equation}
    dan himpunan lokasi penetrasi $l^m_{j,k}$ dapat didefinisikan sebagai
    \begin{equation}\label{lpv2}
    %\mathit{L^m_j}=\{l^m_{j,k}\ |\ m\in M\wedge k\in K\wedge j\in \Z^+\}.
    \mathit{L^m_j}=\{l^m_{j,1},l^m_{j,2},...,l^m_{j,k},...,l^m_{j,w}\}.
    \end{equation}
    \item \textit{Daya keluaran PLTS}: Himpunan daya-daya keluaran PLTS untuk suatu rentang waktu pada skenario penetrasi ke-$m$ dan proses penetrasi ke-$k$ didefinisikan sebagai daya keluaran PLTS $e^m_{k}$. Untuk skenario penetrasi ke-$m$, daya-daya keluaran PLTS diformulasikan sebagai
    \begin{equation}\label{dkp1}
        \mathit{E^m}=\{E^m_{1},E^m_{2},...,E^m_{k},...,E^m_{w}\}.
    \end{equation}
    dengan daya keluaran PLTS $E^m_{k}$ dideskripsikan sebagai
    \begin{equation}\label{dkp2}
        \mathit{E^m_k}=\{e^m_{k,1},e^m_{k,2},...,e^m_{k,t},...,e^m_{w,v}\}.
    \end{equation}
    \item \textit{Permintaan beban}: Permintaan beban $g^m_{k}$ dideskripsikan sebagai himpunan permintaan-permintaan beban untuk suatu rentang waktu pada skenario penetrasi ke-$m$ dan proses penetrasi ke-$k$. Untuk skenario penetrasi ke-$m$, permintaan-permintaan beban dirumuskan sebagai
    \begin{equation}\label{pb1}
        \mathit{G^m}=\{G^m_{1},G^m_{2},...,G^m_{k},...,G^m_{w}\}.
    \end{equation}
    dengan permintaan beban $G^m_{k}$ diformulasikan sebagai
    \begin{equation}\label{pb2}
        \mathit{G^m_k}=\{g^m_{k,1},g^m_{k,2},...,g^m_{k,t},...,g^m_{w,v}\}.
    \end{equation}
    
    \end{enumerate}
    
\section{Algoritme}
Dengan menggunakan diagram alir penentuan kapasitas \textit{hosting} stokastik untuk PLTS tersebar yang ditunjukkan pada Gambar \ref{fcmcmc}, dan dengan skenario-skenario penetrasi PLTS yang telah dirancang, Algoritme \ref{alg:marchai1} dirancang.

%\subsubsection{Perancangan Markov Chain Monte Carlo}
%\subsubsection{Pembentukan Algoritme}
\bgroup
\begin{algorithm}[htbp]
	\caption{Markov Chain Monte Carlo untuk Penentuan Kapasitas \textit{Hosting} Stokastik}\label{alg:marchai1}
	\begin{algorithmic}[1]
		\Procedure{MakeSetPhotovoltaicPenetration}{}
		\State Create sorted data set $\mathit{D}$
		\EndProcedure \\
		
		\Procedure{MakeSetMarkovStates}{}
		\State\textproc {MakeSetPhotovoltaicPenetration}
		\State Create set $Y$ from set $D$
		\EndProcedure \\
		
		\Procedure{ObtainLoadCurve}{}
		\State Obtain load curve containing load demands in time-series
		\EndProcedure \\
		
		\Procedure{ObtainPhotovoltaicOutput}{}
		\State Obtain photovoltaic output in time-series
		\EndProcedure \\	
		
		\Procedure{MakeSetPhotovoltaicAllocation}{}
		\State Create set $A$ using Eq. (\ref{pas1})
		\State Create set $A^m$ using Eq. (\ref{pas2})
		\EndProcedure \\
		
		\Procedure {MakeSetCustomerWithPhotovoltaic}{}
		\State Create $\mathit{C^m_j}$ using Eq. (\ref{cpv1})
		\EndProcedure\\
		
		\Procedure{MakeSetPhotovoltaicSize}{}
		\State Create $\mathit{X^m_j}$ using Eq. (\ref{xpv2})
		\EndProcedure\\

		\Function{CalculateCP}{}
		\State Calculate $cp^m_k$ using Eq. (\ref{cpl1})
		\State\Return $cp^m_k$
		\EndFunction \\
		
		\Function{CalculateXP}{}
		\State Calculate $xp^m_k$ using Eq. (\ref{xpl1})
		\State\Return $xp^m_k$
		\EndFunction \\
		
		\Procedure{MakeSetIntegrationLocation}{}
		\State  Create $\mathit{L^m_j}$ using Eq. (\ref{lpv2})
		\EndProcedure

		\algstore{bkbreak}
	\end{algorithmic}
\end{algorithm}
\addtocounter{algorithm}{-1}
\begin{algorithm}[htbp]
	\caption{Markov Chain Monte Carlo untuk Penentuan Kapasitas \textit{Hosting} Stokastik}
	\begin{algorithmic}[1]
	\algrestore{bkbreak}
		\Function{CalculateH}{}
		\State Calculate $\mathit{h^m}$ using Eq. (\ref{phc1})
		\State\Return $\mathit{h^m}$
		\EndFunction\\

		\Function {TransProbMatrix}{}
		\State\textproc {MakeSetMarkovStates}		
		\For {each $\{y_i,y_j\}$ in $Y$}
		\State $y_i \gets$ prior outcome value
		\State $y_j \gets$ posterior outcome value
		\State Calculate $p_{ij}$ using Eq. (\ref{tp2})
		\EndFor 		
		\State Create $r\times r$ transition probability matrix $\mathbf{P}$ with $p_{ij}$ as entries
		\State\Return matrix $\mathbf{P}$
		\EndFunction\\

		\Function{PerformPhotovoltaicAllocation}{}
		\If {$r$ changes}
		\State Update $Y$ with remaining $y$
		\EndIf
		\For {each $k$ in rooftop photovoltaic process $a^m_k$}
		\State\textproc {TransProbMatrix}($a^m_k$)
		\State Calculate state $y_j$ given state $y_i$ using matrix $\mathbf{P}$
		\EndFor
		\State\Return State $y_j$
		\EndFunction\\
		
		\Function{PerformMarkovChainsMonteCarlo}{}
		\State\textproc{ObtainLoadCurve}
		\State\textproc{ObtainPhotovoltaicOutput}
		\State Construct a code of the distribution grid under study
		\State Define the operational limits concerned
		\State Initialize  $m\ :\ m=1$
		\State Initialize  $k\ :\ k=1$
		\State Obtain $a^m_k$
		\State\textproc {MakeSetPhotovoltaicAllocation}
		\State Initialize $y_j$
		\State Obtain $c^m_{j,k}$, $x^m_{j,k}$, and $l^m_{j,k}$
		\State\textproc {MakeSetCustomerWithPhotovoltaic}
		\State\textproc {MakeSetPhotovoltaicSize}
		\State\textproc {MakeSetIntegrationLocation}
		\While{$m$ in $cp^m_k \neq$ $u$}
		\State\textproc{CalculateCP}
		\State\textproc{CalculateXP}
		\algstore{bkbreak}
	\end{algorithmic}
\end{algorithm}	
\addtocounter{algorithm}{-1}
\begin{algorithm}[!t]
	\caption{Markov Chain Monte Carlo untuk Penentuan Kapasitas \textit{Hosting} Stokastik}
	\begin{algorithmic}[1]
	\algrestore{bkbreak}
		\If{$m$ $>=$ 2}
		\State Update $a^m_k$
		\EndIf	

		\While {$k$ in $cp^m_k \neq$ $w$}		
		\If{$k$ $>=$ 2}	
		\State $y_i \gets$ $y_j$

		\State\textproc {PerformPhotovoltaicAllocation}
		\State Determine $c^m_{j,k}$, $x^m_{j,k}$, and $l^m_{j,k}$ 
		\EndIf	
		\State Penetrate rooftop photovoltaic using $c^m_{j,k}$, $x^m_{j,k}$, and $l^m_{j,k}$
		\State Run power flow in time-series
		\State Evaluate the results using the operational limits
		\If {violation occurs at $\mathit{xp^m_k}$}
		\State Determine $\mathit{xp^m_{k-\text{1}}}$ as rooftop photovoltaic hosting capacity
		\EndIf		
		\State Update $k\ :\ k=k+1$
		\EndWhile
		\State Calculate $\mathit{h_{min}}$ using Eq. (\ref{phc2})
		\State Update $m\ :\ m=m+1$
		\EndWhile		
		\State\Return $\mathit{A^m}$, $\mathit{C^m_j}$, $\mathit{X^m_j}$, $\mathit{CP^m}$, $\mathit{XP^m}$, $\mathit{L^m_j}$, and $\mathit{h_{min}}$
		\State Record the desired results
		\EndFunction
	\end{algorithmic}
\end{algorithm}
\egroup

%\subsection{Pengembangan Metode dengan Pertimbangan PLTS-Baterai Tersebar}
%\section{Penganalisisan Hasil dan Penarikan Kesimpulan}