{\setstretch{1}
Abstrak ditulis dalam Bahasa Indonesia. Penetrasi pembangkit listrik tenaga surya pada distribusi tegangan rendah telah menjadi topik utama dalam dunia energi terbarukan. Penelitian ini membahas tentang potensi energi surya dan keuntungan dari pembangkit listrik tenaga surya (PLTS) pada distribusi tegangan rendah. Penelitian ini juga membahas tentang teknologi yang digunakan untuk memasang PLTS pada distribusi tegangan rendah dan dampaknya terhadap sistem kelistrikan. Dari hasil penelitian, ditemukan bahwa PLTS dapat menghasilkan listrik secara efisien dan dapat membantu dalam meningkatkan ketersediaan listrik di daerah yang sulit dijangkau oleh jaringan distribusi listrik utama. Namun, untuk memaksimalkan manfaat PLTS, dibutuhkan investasi dan perencanaan yang matang. Penelitian ini dapat menjadi referensi bagi pemerintah dan pelaku industri dalam mengembangkan energi terbarukan dan mencapai target emisi karbon yang lebih rendah.

\noindent\textbf{Kata kunci---}Energi terbarukan, penetrasi PLTS, sistem istribusi tegangan rendah.
}
% Maksimal 500 kata