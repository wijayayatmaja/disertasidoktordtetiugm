\chapter{PENDAHULUAN}

\section{Latar Belakang}
Penetrasi pembangkit listrik tenaga surya (PLTS) tersebar pada jaringan distribusi tenaga listrik yang semakin tinggi \cite{IEApvps2020,irena2020} mendorong perlunya investigasi terhadap dampak negatif yang diterima oleh sistem distribusi tegangan rendah \cite{Hasheminamin2015,Reno2016,Olivier2016,Santos-Martin2016,Elrayyah2017,Mortazavi2015,Andresen2016}. Kenaikan penetrasi tersebut disebabkan oleh harga unit PLTS yang semakin murah, visi untuk mengurangi pembangkit berbahan bakar fosil dan CO$_2$, perkembangan teknologi PLTS, serta dukungan dari regulasi pemerintah \cite{Abreu2019,Comello2018,Karakaya2015,Handayani2019}. Pada level tertentu, penetrasi PLTS tersebar berpotensi menyebabkan masalah pada kinerja operasi sistem distribusi tenaga listrik, seperti tegangan lebih \cite{Hasheminamin2015,Reno2016,Olivier2016,Santos-Martin2016,Elrayyah2017}, aliran daya balik \cite{Mortazavi2015,Hasheminamin2015}, pelanggaran kapasitas termal konduktor \cite{Reno2016,Andresen2016}, dan lain-lain. Untuk mencegah terjadinya permasalahan-permasalahan tersebut, penentuan kapasitas \textit{hosting} sistem distribusi dalam menerima penetrasi PLTS tersebar perlu dilakukan. Kapasitas \textit{hosting} PLTS didefinisikan sebagai jumlah daya maksimal dari penetrasi PLTS tersebar yang diintegrasikan pada jaringan distribusi sebelum terjadi masalah atau gangguan operasi pada jaringan tersebut \cite{Bollen2011}. 

Kapasitas \textit{hosting} adalah pertimbangan utama dalam perencanaan sistem distribusi tenaga listrik untuk menghadapi kenaikan penetrasi PLTS tersebar\cite{Dubey2017,Epri2012}. Terkait hal itu, akurasi dari penentuan kapasitas \textit{hosting} menjadi perhatian utama. Dari tahun ke tahun, penelitian-penelitian sebelumnya melakukan pengembangan metode penentuan kapasitas \textit{hosting} untuk meningkatkan akurasi terhadap karakteristik natural dari penetrasi PLTS \cite{Breker2015,Kolenc2015,Bollen2017,Dubey2017,Epri2012}. Berkaitan dengan peningkatan akurasi tersebut, tantangan utama dalam penentuan kapasitas \textit{hosting} untuk penetrasi PLTS tersebar adalah pemodelan penetrasi tersebar yang melibatkan karakteristik ketidakpastian dan variabilitas natural dari penetrasi PLTS \cite{Heslop2016,Luthander2017,Emmanuel2017,Breker2015,Kolenc2015,Bollen2017,Dubey2017,Epri2012}.

Dalam mempertimbangkan karakteristik ketidakpastian dan variabilitas penetrasi PLTS tersebut, serta dalam usaha untuk mendapatkan hasil kapasitas \textit{hosting} yang akurat, dua pendekatan yang paling banyak digunakan adalah pendekatan \textit{deterministic} dan pendekatan stokastik. Pendekatan \textit{deterministic} untuk penentuan kapasitas \textit{hosting} digunakan pada \cite{Conti2007,Shayani2011,Kabir2016,Heslop2016,Luthander2017,Emmanuel2017,Abdelkader2020}. Pada pendekatan ini, kapasitas \textit{hosting} dicari dengan menetapkan nilai-nilai tertentu sebagai variabel-variabel simulasi. Proses penyebaran PLTS pada pendekatan \textit{deterministic} dilakukan dengan menetapkan ukuran daya PLTS dengan suatu nilai tertentu dan lokasi integrasi PLTS pada satu atau beberapa \textit{bus} dalam sistem-sistem distribusi yang digunakan. Oleh sebab itu, pendekatan \textit{deterministic} umumnya digunakan dalam menentukan kapasitas \textit{hosting} pada satu atau beberapa PLTS tersebar yang berukuran daya besar \cite{Wang2004,Gozel2009}. Penggunaan skenario \textit{worst-case} dan pembulatan nilai-nilai yang ekstrim dilakukan pada pendekatan \textit{deterministic} tersebut \cite{Mokryani2017}. Selain itu, pendekatan ini tidak memasukkan ketidakpastian-ketidakpastian ukuran daya PLTS dan lokasi integrasi PLTS sehingga karakteristik acak dari penetrasi tidak terwakilkan \cite{Conti2007,Kabir2016}. Oleh karena itu, pendekatan \textit{deterministic} tidak cukup untuk dipakai pada penentuan dan analisis penetrasi PLTS tersebar \cite{Zubo2017}. 

Selain pendekatan \textit{deterministic} tersebut, pendekatan lain yang dikembangkan dalam penentuan kapasitas \textit{hosting} PLTS tersebar adalah pendekatan stokastik \cite{Breker2015,Kolenc2015,Bollen2017}. Dalam pendekatan stokastik, perhitungan kapasitas \textit{hosting} bergantung pada kondisi acak dari persebaran PLTS pada jaringan distribusi \cite{Breker2015}. Oleh karena itu, metode-metode yang dikembangkan dengan basis pendekatan stokastik ditandai dengan dipertimbangkannya karakteristik-karakteristik ketidakpastian dan variabilitas sebagai masukan dalam pemodelan penetrasi PLTS tersebar. Dalam hal ini, Monte Carlo adalah metode yang paling banyak digunakan untuk menyimulasikan ketidakpastian dan variabilitas dari penetrasi PLTS tersebar secara stokastik \cite{Torquato2018,Gooding2014,Dubey2017,Wang2020,Bollen2017,Reno2016,Epri2012,Kolenc2015}. Karakteristik ketidakpastian dari penetrasi PLTS tersebar antara lain ketidakpastian ukuran daya dan lokasi integrasi PLTS tersebar pada jaringan distribusi tegangan rendah \cite{Dubey2015,Epri2012,Ding2017,Dubey2017}. %Selain itu, terdapat juga ketidakpastian berupa probabilitas terkait dengan urutan pelanggan dalam menghubungkan PLTS-nya pada jaringan distribusi tenaga listrik. Probabilitas-probabilitas tersebut menyebabkan terjadinya pola acak natural dari disribusi PLTS tersebar pada sepanjang saluran sistem distribusi tegangan rendah \cite{Dubey2015,Epri2012,Ding2017,Dubey2017}. 
Karakteristik lainnya adalah karakteristik variabilitas. Variabilitas terdapat pada produksi daya keluaran PLTS dan permintaan beban. Ketika diamati dalam kerangka acuan runtun waktu, variabilitas daya keluaran PLTS dan permintaan beban ini berwujud fluktuasi yang bervariasi secara tidak tentu \cite{Behravesh2018,Wang2020}. Dengan penjelasan-penjelasan tersebut, dapat dipahami bahwa metode penentuan kapasitas \textit{hosting} stokastik diperlukan untuk mempertimbangkan karakteristik-karakteristik ketidakpastian penetrasi tersebut agar dapat dihasilkan model penetrasi PLTS tersebar yang representatif \cite{Gooding2014,Dubey2017,Wang2020}.

Meskipun pendekatan stokastik yang telah dikembangkan penelitian-penelitian sebelumnya telah berhasil merepresentasikan persebaran acak pada \textit{bus-bus} dalam jaringan distribusi, tetapi ketidakpastian lokasi dan ukuran daya PLTS pada sisi sekunder transformator jaringan tegangan rendah (JTR) belum dipertimbangkan. Penggunaan \textit{lumped load} pada sisi sekunder transformator JTR mengabaikan parameter jarak dan impedansi antar rumah pelanggan \cite{Ding2017,Atmaja2019a,Atmaja2019b,Atmaja2020a}. Konsekuensinya, akurasi berupa representasi riil dari penetrasi PLTS terkait ketidakpastian lokasi dan ukuran daya PLTS belum dapat dihasilkan. Selanjutnya, pelanggan-pelanggan individual tersebut mempunyai karakteristik masing-masing yang dibedakan berdasarkan jenis/kategori pelanggan. Berbagai macam kategori pelanggan, seperti pelanggan rumah tangga, komersial, pendidikan, industri, dan militer, mempunyai karakteristik yang berbeda sehingga kemungkinan masing-masing untuk memasang PLTS juga berbeda-beda. Penelitian-penelitian pada \cite{Epri2012,Dubey2015} menggunakan kategori pelanggan untuk membedakan antara pelanggan dengan daya PLTS kecil (pelanggan rumah tangga) dan pelanggan dengan daya PLTS besar (pelanggan komersial). Dalam penelitian-penelitian tersebut, probabilitas kemunculan pelanggan dengan PLTS untuk masing-masing kategori tidak dipertimbangkan. Dengan kata lain, probabilitas terdapatnya daya PLTS pelanggan komersial yang sama besarnya atau lebih kecil daripada daya PLTS pelanggan rumah tangga diabaikan. Akibatnya, representasi riil dari penetrasi PLTS tidak dapat dicapai. Sementara itu, penelitian-penelitian tersebut hanya terbatas pada kategori pelanggan rumah tangga dan pelanggan komersial. Kategori pelanggan untuk sektor pendidikan, industri, dan militer tidak dipertimbangkan. 

Selain ketidakpastian penetrasi PLTS tersebut, hal lain yang perlu ditingkatkan adalah akurasi metode terkait variabilitas permintaan beban dan daya keluaran PLTS. Untuk mengamati dampak-dampak penetrasi PLTS pada sistem distribusi tegangan rendah, penelitian pada \cite{Dubey2017} menggunakan resolusi pengamatan per 1 jam. Hal ini menyebabkan variabilitas permintaan beban dan daya keluaran PLTS pada sampel waktu dibawah 1 jam tidak dapat terekam. Selanjutnya, walaupun menggunakan resolusi pengamatan lebih tinggi, yaitu per 15 menit, penelitian pada \cite{Torquato2018} menggunakan satu kurva permintaan beban dan daya keluaran PLTS. Hal ini tentu mengabaikan variabilitas-variabilitas pada kondisi-kondisi yang lain, misal kurva permintaan beban yang berbeda-beda pada hari yang berbeda atau kurva daya keluaran PLTS yang berbeda-beda pada kondisi awan/musim yang berbeda. Terkait hal tersebut, masalah yang muncul adalah beban komputasi. Meningkatkan resolusi pengamatan menjadi lebih tinggi adalah solusi untuk peningkatan akurasi, tetapi hal itu akan meningkatkan beban komputasi \cite{Beck2016}.

Permasalahan selanjutnya yang belum terselesaikan pada penelitian-penelitian sebelumnya adalah karakteristik ketidakpastian yang terkait dengan kategori penetrasi. Kategori penetrasi didefinisikan sebagai kategori dari komponen pembangkit yang dipenetrasikan, PLTS saja atau PLTS yang terintegrasi dengan baterai (PLTS-baterai). Dalam hal ini, baterai-baterai yang dipasang pada suatu sistem distribusi tenaga listrik yang telah terintegrasi pada PLTS tersebar tersebar terbukti dapat mengurangi dampak negatif penetrasi PLTS tersebar terhadap jaringan distribusi tegangan rendah \cite{Mokhtari2013,Alam2013,Tran2019}. Strategi-strategi manajemen sistem penyimpan energi telah dirancang untuk mengendalikan operasi baterai-baterai yang dipasang tersebut. Strategi-strategi tersebut diarahkan untuk bisa digunakan baik pada sistem yang terhubung grid maupun yang beroperasi secara \textit{islanded} \cite{Tran2019}. Untuk tujuan tersebut, salah satu cara yang dilakukan adalah pendekatan dalam mengoordinasikan baterai yang terintegrasi pada PLTS tersebar dari para pelanggan dengan kendali penanggulangan kenaikan tegangan \cite{Mokhtari2013}. Selanjutnya, pengurangan dampak dari perubahan daya keluaran PLTS tersebar yang mendadak dengan mengimplementasikan \textit{discharge} jangka pendek terbukti juga dapat dilakukan \cite{Alam2013}. Meskipun penelitian tentang strategi-strategi manajemen sistem penyimpan energi tersebut telah dikerjakan, pertimbangan kemungkinan pelanggan-pelanggan yang mengintegrasikan panel PLTS sekaligus dengan baterai (PLTS-baterai) pada sistem distribusi belum diteliti. Penelitian sebelumnya hanya memasukkan baterai-baterai secara terpisah setelah simulasi penetrasi PLTS tersebar selesai dilakukan. Salah satu urgensi untuk mempertimbangkan PLTS-baterai dalam perhitungan kapasitas \textit{hosting} tersebut adalah semakin meningkatnya jumlah penetrasi PLTS-baterai. Di Jerman tercatat 40\% penetrasi yang terjadi adalah penetrasi PLTS-baterai \cite{irena2019}. Berkaitan dengan penyelesaian permasalahan-permasalahan tersebut, pendekatan Monte Carlo mempunyai beberapa keterbatasan. Monte Carlo mempunyai keterbatasan terkait akurasi dan beban komputasi. Pendekatan Monte Carlo merupakan pendekatan yang paling cepat perkembangannya dan paling banyak dipakai saat ini dalam penentuan kapasitas \textit{hosting} stokastik. Meskipun demikian, Monte Carlo memerlukan jumlah simulasi yang semakin banyak untuk mendapatkan akurasi yang tinggi \cite{Dubey2017}, padahal penambahan jenis karakteristik ketidakpastian juga akan meningkatkan jumlah simulasi Monte Carlo.

%Karakteristik ketiga dari penetrasi PLTS yang perlu dikembangkan adalah karakteristik yang berkaitan dengan kategori daya pelanggan. Penelitian-penelitian \cite{Epri2012,Dubey2015,Dubey2017,Ding2017} menggunakan himpunan ukuran-ukuran daya PLTS yang didapat data historis untuk dijadikan sampel masukan simulasi Monte Carlo. Dengan begitu, penelitian-penelitian tersebut mempertimbangkan ukuran-ukuran daya riil dari PLTS. Namun, probabilitas munculnya kejadian kategori ukuran daya PLTS tertentu pada proses penetrasi tidak dipertimbangkan. Kaitannya dengan bahasan sebelumnya, kategori daya pelanggan, selain membedakan dari sisi kategori pelanggan juga perlu untuk membedakan dari sisi kategori penetrasi. Artinya, pelanggan-pelanggan dengan PLTS-baterai juga dipertimbangkan.

%Selanjutnya, kategori pelanggan, kategori penetrasi, dan kategori daya penetrasi adalah kategori-kategori yang didapatkan dari data historis. Oleh karena itu, untuk dapat memasukkan kategori-kategori tersebut ke dalam model estimasi penetrasi PLTS atap, suatu metode yang berbasis pada data historis diperlukan. Terkait implikasinya, studi untuk pengembangan metode tersebut mempunyai tingkat urgensi yang tinggi. Hal ini ditunjukkan dari kebutuhan untuk menyediakan pertimbangan yang akurat dan representatif pada perencanaan sistem distribusi tenaga listrik dalam menghadapi penetrasi PLTS atap yang semakin meningkat, serta dalam upaya mengantisipasi dampak negatif dari penetrasinya.

\section{Perumusan Masalah}
Permasalahan terkait penentuan kapasitas \textit{hosting} stokastik dalam penelitian disertasi ini dirumuskan sebagai:
\begin{enumerate}
    \item Ketidakpastian lokasi dan ukuran daya PLTS, serta variabilitas permintaan beban dan iradiasi matahari dalam runtun waktu.
    \item Ketidakpastian lokasi dan ukuran daya PLTS/PLTS-baterai, serta operasi baterai pada penetrasi PLTS-baterai.
\end{enumerate}

% \begin{enumerate}
% \item Kurangnya representasi ketidakpastian penetrasi yang disebabkan karena penggunaan \textit{lumped load}.
% \item Tidak diperhitungkannya probabilitas kemunculan jenis/kategori pelanggan.
% \item Kurangnya representasi dari kriteria variabilitas karena kerangka runtun waktu yang disederhanakan.
% \item Semakin tingginya beban komputasi karena semakin tingginya resolusi pengamatan.
% \item Tidak dipertimbangkannya probabilitas kemunculan pelanggan dengan PLTS-baterai.
% \item Perlunya investigasi dampak dari peningkatan penetrasi PLTS-baterai pada jaringan distribusi.
% \end{enumerate}

Masalah-masalah tersebut bersumber pada kebutuhan mengenai peningkatan akurasi dari metode penentuan kapasitas \textit{hosting} stokastik. Dalam hal ini, metode yang akurat diperlukan untuk meminimalkan kesalahan pada pengambilan keputusan dalam menghadapi kenaikan penetrasi PLTS dan PLTS-baterai, serta dalam menentukan strategi pencegahan dampak negatif penetrasi pada jaringan distribusi.

\section{Tujuan Penelitian}
Penelitian disertasi ini bertujuan untuk meningkatkan akurasi dengan model yang representatif terhadap kondisi riil penetrasi PLTS dan PLTS-baterai dengan mengembangkan metode penentuan kapasitas \textit{hosting} stokastik berbasis pendekatan Markov Chain Monte Carlo. Hal itu diwujudkan melalui tujuan spesifik:
\begin{enumerate}
    \item Untuk meningkatkan akurasi metode terkait ketidakpastian penetrasi PLTS. Tujuan ini dirumuskan untuk menyelesaikan permasalahan kurangnya representasi ketidakpastian penetrasi karena penggunaan \textit{lumped load} dan tidak diperhitungkannya probabilitas kemunculan jenis/kategori pelanggan.
    \item Untuk meningkatkan akurasi metode terkait variabilitas permintaan beban dan daya keluaran PLTS dalam runtun waktu. Tujuan ini diformulasikan untuk menyelesaikan permasalahan kurangnya representasi dari kriteria variabilitas karena kerangka runtun waktu yang disederhanakan dan semakin tingginya beban komputasi karena semakin tingginya resolusi pengamatan dari runtun waktu tersebut.
    \item Untuk meningkatkan akurasi metode terkait ketidakpastian dan variabilitas penetrasi PLTS/PLTS-baterai, serta melakukan investigasi dampak dari peningkatan penetrasi PLTS-baterai terhadap jaringan distribusi. Tujuan ini disusun untuk menyelesaikan permasalahan tidak dipertimbangkannya probabilitas kemunculan pelanggan dengan PLTS-baterai dan perlunya investigasi dampak dari peningkatan penetrasi PLTS-baterai pada jaringan distribusi tegangan rendah.
\end{enumerate}


\section{Manfaat Penelitian}
Penelitian ini dapat digunakan oleh perencana dan operator sistem distribusi sebagai salah satu pertimbangan untuk menghadapi dampak negatif dari kenaikan yang tinggi dari penetrasi PLTS-baterai pada jaringan distribusi tegangan rendah. Selain itu, untuk peneliti-peneliti dibidang integrasi PLTS-baterai pada jaringan distribusi tegangan rendah, penelitian ini dapat dijadikan sebagai sumber acuan. 


%Pada skala tertentu, penetrasi PLTS tersebar pada jaringan distribusi tegangan rendah dapat menyebabkan gangguan pada performa operasinya, seperti tegangan lebih, aliran daya balik, pelanggaran kapasitas termal konduktor dan lain-lain. Terkait hal itu, metode penentuan kapasitas \textit{hosting} yang representatif dengan kondisi riil penetrasi PLTS tersebar diperlukan. Meskipun metode penentuan kapasitas \textit{hosting} stokastik dapat merepresentasikan karakteristik-karakteristik ketidakpastian dengan lebih baik dibandingkan metode berbasis \textit{deterministic}, karakteristik-karakteristik PLTS-baterai belum dipertimbangkan pada penelitian-penelitian sebelumnya. Kenaikan penetrasi PLTS-baterai mendorong urgensi untuk melibatkan PLTS-baterai dalam penghitungan kapasitas \textit{hosting} stokastik. 

%Sementara itu, karena metode berbasis stokastik melibatkan banyak proses acak, evaluasi dari akurasi metode diperlukan. Berkaitan dengan hal tersebut, performa akurasi dari estimasi kapasitas \textit{hosting} berbanding lurus dengan jumlah simulasi Monte Carlo yang dilakukan. Jumlah simulasi yang banyak diperlukan untuk mendapatkan akurasi yang tinggi, padahal jumlah simulasi yang semakin banyak memberikan beban komputasi yang semakin berat. Pada prakteknya, beban komputasi yang berat dapat berakibat pada lamanya waktu pengambilan keputusan oleh operator distribusi tenaga listrik terhadap setiap permintaan penyambungan PLTS tersebar pada jaringan distribusi. Dalam hal ini, keputusan yang cepat dibutuhkan untuk dapat mengatasi jumlah permintaan penyambungan PLTS dari konsumen yang semakin meningkat. Maka dari itu, metode penentuan kapasitas \textit{hosting} stokastik perlu dikembangkan untuk mempertimbangkan karakteristik-karakteristik ketidakpastian PLTS-baterai, serta untuk meningkatkan akurasi tanpa meningkatkan beban komputasi. 

%Meskipun mempunyai akurasi yang baik dan waktu komputasi yang cepat pada sistem tipikal, kompleksitas dan beban komputasi dari metode penentuan kapasitas \textit{hosting} berbasis \textit{deterministic} meningkat jika diterapkan pada sistem yang lebih besar. Dalam hal ini, karena kemampuannya dalam menghadapi sistem yang besar dan rumit, serta kemampuannya dalam merepresentasikan karakteristik-karakteristik acak penetrasi PLTS tersebar, metode penentuan kapasitas \textit{hosting} stokastik dikembangkan. 

%Metode penyebaran PLTS skala besar berbasis proses Monte Carlo telah terbukti mampu mempertimbangkan ketidakpastian profil beban, daya keluaran PLTS, pelanggan yang melakukan penyambungan PLTS, lokasi penetrasi, dan ukuran PLTS secara terpisah. Meskipun demikian, penelitan yang memasukkan variabel-variabel ketidakpastian secara bersamaan dalam suatu metode penentuan kapasitas \textit{hosting} PLTS perlu untuk diteliti. Hal ini berkaitan dengan usaha untuk menyediakan model penyebaran yang representatif. Salah satu kendalanya, pelibatan variabel-variabel ketidakpastian secara bersamaan akan meningkatkan komplektsitas perhitungan dan waktu komputasi.

%Selanjutnya, penelitian-penelitian sebelumnya telah berhasil menunjukkan bahwa pemasangan baterai tersebar pada jaringan tenaga listrik mampu mengurangi dampak penetrasi PLTS skala besar. Meskipun demikian, model penetrasi PLTS-PLTS tersebar yang terintegrasi dengan baterai belum disediakan. Karena baterai telah terbukti dapat meningkatkan kompleksitas aliran daya dalam jaringan tenaga listrik, pelibatan baterai yang terintegrasi pada PLTS dalam penentuan kapasitas \textit{hosting} perlu diteliti. Dalam hal ini, tantangan yang dihadapi juga berkaitan dengan penyediaan metode penentuan kapasitas \textit{hosting} PLTS yang representatif dan akurat tetapi tidak menaikkan waktu komputasi. 

%\section{Keaslian Penelitian}
%Penjelasan mengenai keaslian penelitian disampaikan melalui perbandingan dari penelitian-penelitian sebelumnya dengan penelitian yang diusulkan. Poin-poin penting yang didapat dalam perbandingan tersebut dirangkum dalam Tabel \ref{tab-novelty}. 
%\bgroup
%%\def\arraystretch{1.3}
%{\renewcommand{\arraystretch}{1.3}
%\begin{sidewaystable}[htbp]
%	\caption{Perbandingan Penelitian-Penelitian yang Berkaitan Dengan Topik Penelitian yang Diusulkan}
%	\vspace{-12pt}
%	\begin{center}
%		%\setlength\tabcolsep{4pt}
%		\begin{tabular}{|@{\hspace*{0.7em}\extracolsep{\fill}}p{10em}@{\hspace*{0.7em}\extracolsep{\fill}}|@{\hspace*{0.7em}\extracolsep{\fill}}p{3em}@{\hspace*{0.7em}\extracolsep{\fill}}|@{\hspace*{0.7em}\extracolsep{\fill}}p{5em}@{\hspace*{0.7em}\extracolsep{\fill}}|@{\hspace*{0.7em}\extracolsep{\fill}}p{3.1em}@{\hspace*{0.7em}\extracolsep{\fill}}|@{\hspace*{0.7em}\extracolsep{\fill}}p{4em}@{\hspace*{0.7em}\extracolsep{\fill}}|@{\hspace*{0.7em}\extracolsep{\fill}}p{3em}@{\hspace*{0.7em}\extracolsep{\fill}}|}
%			\cline{1-6} 
%			\textbf{\textit{Referensi}}&	
%			%\textbf{\textit{Metode Penetrasi}}& 	
%			%\textbf{\textit{Model Stokastik Keluaran PLTS}}&	
%			%\textbf{\textit{Model Stokastik Profil Beban}}&	
%			\textbf{\textit{Lokasi PLTS Acak}}&	
%			\textbf{\textit{Ukuran Daya PLTS Acak}}&	
%			\textbf{\textit{PLTS- Baterai}}&	
%			\textbf{\textit{Kendali Proses Penetrasi}}&	
%			\textbf{\textit{Analisis Runtun Waktu}}\\
%			\hline A.Dubey (2017) \cite{Dubey2017} 			&Iya	&Iya	&Tidak	&Tidak	&Iya\\
%			\hline F.Ding (2017) \cite{Ding2017}			&Iya	&Iya	&Tidak	&Tidak	&Iya\\
%			\hline M.H.Bollen (2017) \cite{Bollen2017}		&Iya	&Iya	&Tidak	&Tidak	&Iya\\
%			\hline R.Torquato (2018) \cite{Torquato2018}	&Iya	&Tidak	&Tidak	&Tidak	&Iya\\
%			\hline M.Deakin (2019) \cite{Deakin2019}		&Iya	&Iya	&Tidak	&Iya	&Iya\\
%			%\hline S.Wang (2020) \cite{Wang2020}			&Iya	&Iya	&Tidak	&Tidak	&Iya\\
%			\hline P.P.Vergara (2020) \cite{Vergara2020}	&Iya	&Tidak	&Tidak	&Tidak	&Iya\\
%			\hline M.Al-Saffar (2020) \cite{Al-Saffar2020}	&Iya	&Iya	&Iya	&Tidak	&Iya\\
%			\hline Penelitian ini							&Iya	&Iya	&Iya	&Iya	&Iya\\
%			\hline
%		\end{tabular}
%		\label{tab-novelty}
%	\end{center}
%	%\thisfloatpagestyle{floatpage}
%\end{sidewaystable}
%\egroup
%Pada tahun 2017, Anamika Dubey dan Surya Santoso mengusulkan metode penentuan kapasitas \textit{hosting} PLTS tersebar skala besar dengan menggunakan proses stokastik Monte Carlo melalui pendekatan pola sebaran \cite{Dubey2017}, tanpa mempertimbangkan penetrasi PLTS yang terintegrasi dengan baterai. Skenario penempatan lokasi integrasi dan penentuan ukuran daya PLTS tersebar sebanyak 100 dirancang dan disimulasikan. Proses integrasi PLTS tersebar pada jaringan distribusi tegangan rendah dilakukan dengan 50 level penetrasi pelanggan untuk setiap skenario. Perpindahan dari setiap level ke level pelanggan berikutnya menggunakan kenaikan sebesar 2\%. Penentuan ukuran PLTS tersebar yang dipenetrasikan dilakukan dengan mengambil sampel data historis dari \textit{California Distributed Generation Statistics}. Dalam penelitian tersebut digunakan sebuah penyulang riil yang dilayani oleh transformator-transformator 24 MVA. Data historis profil beban dan kurva daya keluaran PLTS tipikal saat cuaca cerah digunakan sebagai input dari metode yang diusulkan. Penelitian tersebut mengusulkan kerangka analisis waktu dengan rentang waktu pengamatan tujuh jam dan resolusi waktu pengamatan satu jam.
%
%Fei Ding dan Barry Mather juga menggunakan metode stokastik Monte Carlo melalui pendekatan pola sebaran seperti pada \cite{Dubey2017}, dengan menerapkan 100 skenario penetrasi dan 50 level penetrasi pelanggan \cite{Ding2017} yang mempertimbangkan lokasi penetrasi dan ukuran PLTS yang acak. Penelitian tersebut juga menggunakan data historis penetrasi PLTS tersebar dari \textit{California Distributed Generation Statistics} pada proses penentuan ukuran daya PLTS tersebar. Pelanggan-pelanggan yang menyambungkan PLTS yang terintegrasi dengan baterai tidak dipertimbangkan. Meskipun demikian, penelitian tersebut mengusulkan metode peningkatan kapasitas \textit{hosting} dengan manajemen sistem distribusi aktif melalui skenario penyerapan daya reaktif oleh PLTS dan optimasi \textit{tap} transformator. Selanjutnya, penelitian tersebut menggunakan 17 penyulang riil sebagai sistem distribusi yang diteliti. Dalam proses analisisnya, peneliti tersebut menggunakan analisis runtun waktu dengan rentang waktu pengamatan empat jam dari pukul 10.00 sampai pukul 14.00. Untuk profil beban dan daya keluaran PLTS, peneliti tersebut menggunakan data historis.
%
%Dengan menggunakan metode stokastik Monte Carlo, Bollen dkk \cite{Bollen2017} mengusulkan metode penentuan kapasitas \textit{hosting} PLTS tanpa mempertimbangkan PLTS yang terintegrasi dengan baterai. Penelitian tersebut melakukan analisis pada runtun waktu tertentu yang merepresentasikan konsumsi beban-beban minimum. Selanjutnya, penelitian tersebut melakukan estimasi terkait model stokastik keluaran PLTS pada rentang tersebut dengan resolusi 10 menit. Kapasitas \textit{hosting} didapat dari hasil evaluasi indeks performa batasan operasional jaringan distribusi tenaga listrik yang dilakukan studi tanpa kendali dalam proses penetrasi.
%
%Selanjutnya, Ricardo Torquato dkk melakukan penelitian mengenai penentuan kapasitas \textit{hosting} PLTS tersebar tanpa baterai yang dilakukan dengan metode stokastik Monte Carlo melalui pendekatan level penetrasi \cite{Torquato2018}. Dalam hal ini, simulasi dilakukan dengan variasi level penetrasi pelanggan dari 20\% sampai 100\%. Untuk setiap penambahan level penetrasi pelanggan, besar kenaikan yang dilakukan adalah 20\%. Kurva profil beban dan daya keluaran PLTS diadapat dari data pengukuran oleh peneliti tersebut. Proses penentuan lokasi penetrasi PLTS tersebar dilakukan secara acak dan berulang dengan kenaikan ukuran daya PLTS sebesar 1 kWp untuk setiap level penetrasi pelanggan. Pada setiap level penambahan ukuran daya PLTS tersebar tersebut, dikerjakan 500 skenario penempatan lokasi di sepanjang penyulang. Penelitian tersebut menggunakan penyulang riil dari jaringan distribusi tegangan rendah di Brazil. Dalam analisisnya digunakan rentang waktu pengamatan dua jam dari pukul 11.00 sampai 13.00 dengan resolusi waktu pengamatan 15 menit.
%
%Matthew Deakin dkk mengembangkan metode stokastik Monte Carlo dalam penentuan kapasitas \textit{hosting} PLTS. Dalam hal ini, penelitian tersebut menentukan kapasitas \textit{hosting} PLTS menggunakan pendekatan \textit{fixed-power} dan \textit{fixed-voltage}. Hasilnya menunjukkan bahwa pendekatan \textit{fixed-voltage} lebih efisien dan membutuhkan waktu komputasi lebih cepat. Peneliti tersebut menggunakan pembatasan proses penetrasi pada pendekatan \textit{fixed-voltage} sehingga tidak semua kemungkinan diamati. Pengamatan hanya dapat dilakukan untuk kasus tertentu terkait dengan tegangan.
%
%Pada tahun 2020, penentuan kapasitas \textit{hosting} PLTS diusulkan dengan basis proses stokastik Monte Carlo oleh Pedro P. Vergara dkk \cite{Vergara2020}. Penelitian tersebut menggunakan data profil beban dan daya keluaran PLTS dari data yang didapat oleh peneliti. Lokasi penetrasi dan ukuran PLTS diacak sampai 100\% penetrasi, kemudian analisis aliran daya berbasis runtun waktu selama satu tahun dikerjakan untuk mengamati dampak terhadap indeks performa operasi yang dipertimbangkan.
%
%Mohammed Al-Safar dkk menggunakan metode stokastik Monte Carlo dalam penetrasi PLTS tersebar. Dalam hal ini, peneliti tersebut fokus pada komunikasi kendali antar baterai pada lokasi lokasi PLTS yang dipenetrasikan. Penelitian tersebut menggunakan 100 skenario dengan mengacak lokasi integrasi dan ukuran PLTS. Prosedur tersebut diulangi sampai 100\% penetrasi dengan kenaikan 10\%. Pada penelitian tersebut, metode peningkatan kapasitas \textit{hosting} diusulkan dengan komunikasi kendali antar baterai.
%
%Pada penelitian ini, Markov Chain Monte Carlo diusulkan dalam penyebaran PLTS. Dalam hal ini, PLTS tersebar yang terintegrasi dengan baterai juga diperhitungkan. Dengan menggunakan matriks probabilitas transisi, integrasi PLTS selanjutnya dikendalikan dengan menggunakan informasi dari integrasi PLTS terakhir \cite{Haggstrom2002,Ching2006}. Matriks probabilitas transisi dibentuk dari \textit{California Distributed Generation Statistics} \cite{CaliforniaDistributedGenerationStatistics}. Karena pengendalian tersebut, berbeda dengan metode stokastik Monte Carlo, Markov Chain Monte Carlo tidak melihat semua kemungkinan penetrasi. Dalam hal ini, penetrasi ke-$p$, bergantung pada penetrasi ke-$(p-1)$. Oleh karena itu, evaluasi dari akurasi metode perlu dilakukan. Studi akurasi dilakukan berdasarkan probabilitas setiap proses penetrasi dalam merasakan batas pelanggaran regulasi operasi \cite{Dubey2017}. Dalam hal ini, profil beban dan daya keluaran PLTS menggunakan resolusi sebesar 15 menit \cite{Beck2016}. Untuk mekanisme penetrasi menggunakan pendekatan pola sebaran dengan cara mengerjakan $n$ skenario penetrasi dan $p$ level penetrasi \cite{Dubey2017,Ding2017}. Setiap skenario penetrasi, lokasi integrasi dan ukuran daya PLTS diacak. Akan tetapi, satu pelanggan dipilih untuk setiap iterasi dalam proses penetrasi \cite{Atmaja2019a,Atmaja2019b,Atmaja2020a}. Penelitian-penelitian pada \cite{Atmaja2019a,Atmaja2019b,Atmaja2020a} adalah penelitian-penelitian yang telah dikerjakan sebelumnya oleh pengusul penelitian ini. Tegangan lebih dipertimbangkan sebagai batas operasional dalam menetapkan kapasitas \textit{hosting} \cite{Dubey2017,Ding2017,Torquato2018}. Untuk analisis aliran daya, penelitian ini menggunakan analisis aliran daya berdasarkan kerangka kerja runtun waktu \cite{Dubey2017,Ding2017,Bollen2017,Torquato2018,Deakin2019,Wang2020,Vergara2020,Al-Saffar2020}. Dalam memodelkan kendali baterai, \textit{state of charge (SOC)} untuk setiap baterai dipertimbangkan \cite{Al-Saffar2020}.
%
