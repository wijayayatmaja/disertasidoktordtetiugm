\section{Transition Probability Matrices}
{\setstretch{1.45}\justifying\setlength{\parindent}{1.25cm}{
Let $P_{ch}$ be the updated transition probability matrix $\mathbf{P}$ for $ch^{th}$ change of customer with rooftop photovoltaic categories $V$, then the transition probability matrices are presented as follows.
\label{appendix1}
\[\mathbf{P}_{ch}=
\begin{blockarray}{ccccc}
	&{2200}&{3500}&{4400}&{5500}\\[-6pt]
	\begin{block}{c[cccc]}
	{2200}&0.2077602594&0.3523426143&0.2244213351&0.2154757911\\[-7pt]
	{3500}&0.1223833168&0.3658788481&0.2618867925&0.2498510427\\[-7pt]
	{4400}&0.1014390305&0.3332996718&0.2855339561&0.2797273416\\[-7pt]
	{5500}&0.1004408858&0.3173364415&0.2811027213&0.3011199514\\[3pt]
 	\end{block}
\end{blockarray}	
\]
\[\mathbf{P}_{ch}=
\begin{blockarray}{ccccc}
	&{2200}&{3500}&{4400}&{5500}\\[-6pt]
	\begin{block}{c[cccc]}
	{2200}&0.2648232611&0.4491163056&0.2860604333&0.0000000000\\[-7pt]
	{3500}&0.1631453535&0.4877415939&0.3491130527&0.0000000000\\[-7pt]
	{4400}&0.1408342096&0.4627409744&0.3964248160&0.0000000000\\[-7pt]
	{5500}&0.1437169168&0.4540642448&0.4022188384&0.0000000000\\[3pt]
 	\end{block}
\end{blockarray}
\]
\[\mathbf{P}_{ch}=
\begin{blockarray}{ccccc}
	&{2200}&{3500}&{4400}&{5500}\\[-6pt]
	\begin{block}{c[cccc]}
	{2200}&0.2678777393&0.4542964245&0.0000000000&0.2778258362\\[-7pt]
	{3500}&0.1658056183&0.4956947584&0.0000000000&0.3384996233\\[-7pt]
	{4400}&0.1419787986&0.4665017668&0.0000000000&0.3915194346\\[-7pt]
	{5500}&0.1397152122&0.4414211194&0.0000000000&0.4188636684\\[3pt]
 	\end{block}
\end{blockarray}
\]
\[\mathbf{P}_{ch}=
\begin{blockarray}{ccccc}
	&{2200}&{3500}&{4400}&{5500}\\[-6pt]
	\begin{block}{c[cccc]}
	{2200}&0.3207872928&0.0000000000&0.3465124309&0.3327002762\\[-7pt]
	{3500}&0.1929967427&0.0000000000&0.4129917314&0.3940115259\\[-7pt]
	{4400}&0.1521508634&0.0000000000&0.4282793093&0.4195698273\\[-7pt]
	{5500}&0.1471308737&0.0000000000&0.4117734392&0.4410956870\\[3pt]
 	\end{block}
\end{blockarray}
\]
\[\mathbf{P}_{ch}=
\begin{blockarray}{ccccc}
	&{2200}&{3500}&{4400}&{5500}\\[-6pt]
	\begin{block}{c[cccc]}
	{2200}&0.3709323218&0.6290676782&0.0000000000&0.0000000000\\[-7pt]
	{3500}&0.2506508298&0.7493491702&0.0000000000&0.0000000000\\[-7pt]
	{4400}&0.2333333333&0.7666666667&0.0000000000&0.0000000000\\[-7pt]
	{5500}&0.2404172732&0.7595827268&0.0000000000&0.0000000000\\[3pt]
 	\end{block}
\end{blockarray}
\]
\[\mathbf{P}_{ch}=
\begin{blockarray}{ccccc}
	&{2200}&{3500}&{4400}&{5500}\\[-6pt]
	\begin{block}{c[cccc]}
	{2200}&0.4807244502&0.0000000000&0.5192755498&0.0000000000\\[-7pt]
	{3500}&0.3184825305&0.0000000000&0.6815174695&0.0000000000\\[-7pt]
	{4400}&0.2621346555&0.0000000000&0.7378653445&0.0000000000\\[-7pt]
	{5500}&0.2632487714&0.0000000000&0.7367512286&0.0000000000\\[3pt]
 	\end{block}
\end{blockarray}
\]
\[\mathbf{P}_{ch}=
\begin{blockarray}{ccccc}
	&{2200}&{3500}&{4400}&{5500}\\[-6pt]
	\begin{block}{c[cccc]}
	{2200}&0.4908850727&0.0000000000&0.0000000000&0.5091149273\\[-7pt]
	{3500}&0.3287802796&0.0000000000&0.0000000000&0.6712197204\\[-7pt]
	{4400}&0.2661279640&0.0000000000&0.0000000000&0.7338720360\\[-7pt]
	{5500}&0.2501261989&0.0000000000&0.0000000000&0.7498738011\\[3pt]
 	\end{block}
\end{blockarray}
\]
\[\mathbf{P}_{ch}=
\begin{blockarray}{ccccc}
	&{2200}&{3500}&{4400}&{5500}\\[-6pt]
	\begin{block}{c[cccc]}
	{2200}&1.0000000000&0.0000000000&0.0000000000&0.0000000000\\[-7pt]
	{3500}&1.0000000000&0.0000000000&0.0000000000&0.0000000000\\[-7pt]
	{4400}&1.0000000000&0.0000000000&0.0000000000&0.0000000000\\[-7pt]
	{5500}&1.0000000000&0.0000000000&0.0000000000&0.0000000000\\[3pt]
 	\end{block}
\end{blockarray}
\]

\newpage
%\section{Sample Python code}
%\begin{lstlisting}[language=Python]
%import numpy as np
%
%def incmatrix(genl1,genl2):
%	m = len(genl1)
%	n = len(genl2)
%	M = None #to become the incidence matrix
%	VT = np.zeros((n*m,1), int)  #dummy variable
%
%	#compute the bitwise xor matrix
%	M1 = bitxormatrix(genl1)
%	M2 = np.triu(bitxormatrix(genl2),1) 
%
%	for i in range(m-1):
%		for j in range(i+1, m):
%			[r,c] = np.where(M2 == M1[i,j])
%			for k in range(len(r)):
%				VT[(i)*n + r[k]] = 1;
%				VT[(i)*n + c[k]] = 1;
%				VT[(j)*n + r[k]] = 1;
%				VT[(j)*n + c[k]] = 1;
%
%	if M is None:
%		M = np.copy(VT)
%	else:
%		M = np.concatenate((M, VT), 1)
%
%	VT = np.zeros((n*m,1), int)
%
%	return M
%\end{lstlisting}
%
%\newpage
%\section{Sample Matlab code}
%\lstinputlisting[language=Octave]{sample/sample-code.m}
%
%\par}\justify}