Puji syukur ke hadirat Allah SWT yang telah melimpahkan rahmat dan barokah-Nya sehingga penulis dapat menyelesaikan tesis dengan judul "Penentuan Kapasitas \textit{Hosting} Stokastik untuk Penetrasi PLTS dan PLTS-Baterai Tersebar pada Jaringan Distribusi". Laporan tesis ini disusun untuk memenuhi salah satu syarat dalam memperoleh gelar \textit{Doctor} (\textit{Dr.}) pada Program Studi Doktor Teknik Elektro Fakultas Teknik Universitas Gadjah Mada Yogyakarta.

Dalam melakukan penelitian dan penyusunan naskah disertasi ini penulis telah mendapatkan banyak dukungan dan bantuan dari berbagai pihak. Penulis mengucapkan terima kasih yang tak terhingga kepada:

\begin{enumerate}
	\item Bapak Ir. Sarjiya, S.T., M.T., Ph.D., IPU. \lipsum[4].
	\item Bapak Ir. Lesnanto Multa Putranto, S.T., M.Eng., Ph.D. IPM. \lipsum[2].
	\item Bapak \lipsum[6].
	\item Ibu \lipsum[5].
\end{enumerate}

Penulis menyadari sepenuhnya bahwa laporan tesis ini masih jauh dari sempurna, untuk itu semua saran, kritik, dan masukan yang bersifat membangun sangat penulis harapkan. Akhir kata, semoga tulisan ini dapat memberikan manfaat dan memberikan wawasan tambahan bagi para pembaca dan khususnya bagi penulis sendiri.

\begin{flushright}
	\begin{tabular}{l}
		Yogyakarta, 20 Maret 2023 \\
		\vspace{1cm} \\
		Wijaya Yudha Atmaja
	\end{tabular}
\end{flushright}

% Untuk menyatakan terima kasih sebaik digunakan kalimat pasif