Puji syukur ke hadirat Allah SWT yang telah melimpahkan rahmat dan barokah-Nya sehingga penulis dapat menyelesaikan tesis dengan judul "Pengembangan Metode Penentuan Kapasitas \textit{Hosting} untuk Penetrasi Pembangkit Listrik Tenaga Surya Atap di Jaringan Distribusi Tegangan Rendah". Laporan tesis ini disusun untuk memenuhi salah satu syarat dalam memperoleh gelar \textit{Master of Engineering} (\textit{M.Eng.}) pada Program Studi Magister Teknik Elektro Fakultas Teknik Universitas Gadjah Mada Yogyakarta.

Dalam melakukan penelitian dan penyusunan laporan tesis ini penulis telah mendapatkan banyak dukungan dan bantuan dari berbagai pihak. Penulis mengucapkan terima kasih yang tak terhingga kepada:

\begin{enumerate}
	\item Bapak Sarjiya, S.T., M.T., Ph.D. sebagai pembimbing utama yang selalu dengan senang hati untuk memberikan nasihat. Ini adalah sebuah pengalaman yang berharga telah dibimbing oleh bapak dan diberi banyak kesempatan terkait penelitian. Terimakasih atas bimbingan dan pengetahuan yang diberikan kepada penulis.
	\item Bapak Lesnanto Multa Putranto, S.T., M.Eng., Ph.D. sebagai pembimbing pendamping yang senantiasa bersedia memberikan bantuan, pengalaman, dan motivasi kepada penulis dalam menghadapi berbagai permasalahan selama penelitian.
	\item Bapak Sarjiya, S.T., M.T., Ph.D. sebagai Ketua Departemen Teknik Elektro dan Teknologi Informasi dan Bapak Dr. Ir. Risanuri Hidayat, M.Sc.  sebagai Ketua Program Studi Magister Teknik Elektro Fakultas Teknik Universitas Gadjah Mada yang memberikan ijin kepada penulis untuk belajar.
	\item Ayah Sugeng Sutarman, Ibu Maulani, Kakak Alam, Kakak Ayu, Kakak Indi, Chelsea, Nora, dan Nara atas motivasi dan dukungan luar biasanya.
	\item Budhe Prof. Darmiyati Zuchdi yang telah memberikan dukungan dalam menyelesaikan studi penulis.
	\item Mas Nandar, Bapak Heru, Bapak Rakhmat, Mas Yaenuri, Mas Nanang, Mas Bolang, Mas Penang, Mas Kris Adi, Eci, Nani, Nurman, Norman, Adhiim, Gigih, Riza, dan Ryanovie yang selalu bersedia meluangkan waktu untuk berdiskusi dengan penulis baik mengenai topik yang bersifat akademis atau nonakademis, baik di Laboratorium Teknik Tenaga Listrik, Laboratorium Instalasi, atau ruang diskusi E2.
	\item Bapak Eka Firmansyah, Bapak Sunu, dan Bapak Galang atas motivasi yang diberikan kepada penulis untuk segera menyelesaikan studi, nasihat, dan diskusi.
	\item Para Dosen Program Studi Magister Teknik Elektro Fakultas Teknik Universitas Gadjah Mada yang telah memberikan bekal ilmu kepada penulis.
	\item Para Karyawan/wati Program Studi Magister Teknik Elektro Fakultas Teknik Universitas Gadjah Mada yang telah membantu penulis dalam proses belajar.
\end{enumerate}

Penulis menyadari sepenuhnya bahwa laporan tesis ini masih jauh dari sempurna, untuk itu semua saran, kritik, dan masukan yang bersifat membangun sangat penulis harapkan. Akhir kata, semoga tulisan ini dapat memberikan manfaat dan memberikan wawasan tambahan bagi para pembaca dan khususnya bagi penulis sendiri.

\begin{flushright}
	\begin{tabular}{c}
		Yogyakarta, 23 Juli 2019 \\
		\vspace{1cm} \\
		Wijaya Yudha Atmaja
	\end{tabular}
\end{flushright}