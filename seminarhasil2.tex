%%%%%%%%%%%%%%%%%%%%%%%%%%%%%%%%%%%%%%%%%%
%%%% Version#01 2023
%%%% Wijaya Y. Atmaja
%%%% wijayayatmaja@gmail.com

%%%% REFERENCES:
%%%% (Pedoman Penulisan DIsertasi FT UGM 2019)
%%%% (Canggih,2018)
%%%% (www.latextemplates.com)
%%%% (stackoverflow.com)

%%%% INSTRUCTION:
%%%% You are given some choices of commands according to the purpose of your manuscript
%%%% Simply remove the % (percent sign) to activate the command you need, and put it back to deactivate the command
%%%% Line with %%%% defines a comment/instruction, keep it as it is

%%%% PURPOSES OF MANUSCRIPT:
%%%% Seminar Hasil 2 (SH2)

%%%%%%%%%%%%%%%%%%%%%%%%%%%%%%%%%%%%%%%%%%

\documentclass{doktordtetiugm}

%%%%--------------------------------------
%%%% Title
%%%%--------------------------------------
\title{
    \vspace{0.15cm} Penentuan Kapasitas \textit{Hosting} Stokastik untuk
    \vspace{0.15cm} Penetrasi PLTS dan PLTS-Baterai Tersebar pada
                    Jaringan Distribusi
}

%%%%--------------------------------------
%%%% Author
%%%%--------------------------------------
\author
    {Wijaya Yudha Atmaja}
    {17/419580/PTK/11690}

%%%%--------------------------------------
%%%% Program Studi
%%%%--------------------------------------
\programstudi{Teknik Elektro}

%%%%--------------------------------------
%%%% Submission Time
%%%%--------------------------------------
\datemonthsubmit{1 November}
\yearsubmit{2023}

\begin{document}

%%%%--------------------------------------
%%%% Hypenation
%%%% You are given 2 options to set hypenation (choose one):
%%%%--------------------------------------
%%%% (Option 1) Use Indonesian hypenation with alteration because hyphenation for languages other than English in LaTeX could be incorrect:
\righthyphenmin=3
\lefthyphenmin=4
\emergencystretch=\maxdimen
\hyphenpenalty=50
\hbadness=10000
\sloppy
%%%% Alteration (correct some words with bad hypenation):
\hyphenation{PLTS me-nge-nai meng-klaim be-be-ra-pa me-ne-rap-kan pe-mo-delan me-re-pre-sen-ta-si-kan men-da-pat-kan}

%%%% (Option 2) Turn off hyphenation (similar to Microsoft Word):
% \righthyphenmin=1000
% \sloppy

%%%%=============================================================
%%%% FRONT MATTER (BAGIAN AWAL)
%%%%=============================================================

%%%%--------------------------------------
%%%% Cover
%%%%--------------------------------------
\printcover{fig/logougm.png}

%%%%--------------------------------------
%%%% Title page
%%%%--------------------------------------
\phantompageref
\addcontentsline{toc}{chapter}{HALAMAN JUDUL}

%%%%--------------------------------------
%%%% Page after title page
%%%%--------------------------------------
\phantomsection
\pageaftercover

%%%%--------------------------------------
%%%% Approval SH2 --> CHOOSE ONE
%%%%--------------------------------------
%%%% Approval page before Seminar Hasil 2 to be signed:
\chapterapproval{contents/approvalpage/approvalbeforeSH2}

%%%% Approval page before Ujian Komprehensif signed:
% \clearpage
% \phantomsection
% \addcontentsline{toc}{chapter}{HALAMAN PERSETUJUAN}
% \phantomsection
% \includepdf[pages={1}]{attachment/signedpromotorSH2.pdf}
% \setcounter{page}{4}

%%%%--------------------------------------
%%%% Table of contents, figures, and tables
%%%%--------------------------------------
\thetoc
\tableofcontents
\let\origaddvspace\addvspace
\renewcommand{\addvspace}[1]{}
\thelof
\listoffigures
\thelot
\listoftables
\renewcommand{\addvspace}[1]{\origaddvspace{#1}}

%%%%--------------------------------------
%%%% List of nomencalture and abbreviation
%%%%--------------------------------------
\chapternomenclature{contents/nomenclature/nomenclature}

%%%%--------------------------------------
%%%% Abstract in Indonesian
%%%%--------------------------------------
\chapterabstrak{contents/abstract/abstrak}

%%%%--------------------------------------
%%%% Abstract in English
%%%%--------------------------------------
\chapterabstract{contents/abstract/abstract}

%%%%=============================================================
%%%% MAIN BODY (BAGIAN UTAMA)
%%%%=============================================================

\startmain
\chapter{PENDAHULUAN}

\section{Latar Belakang}
Euismod elementum nisi quis eleifend quam. Neque viverra justo nec ultrices dui sapien eget. Sagittis vitae et leo duis ut diam \cite{IEApvps2020,irena2019}. Commodo viverra maecenas accumsan lacus vel facilisis. Venenatis urna cursus eget nunc scelerisque viverra. Nulla malesuada pellentesque elit eget gravida cum. Egestas sed sed risus pretium quam vulputate dignissim suspendisse. Risus in hendrerit gravida rutrum quisque non tellus orci \cite{Wang2004,Gozel2009}. A pellentesque sit amet porttitor eget dolor morbi non arcu. Urna cursus eget nunc scelerisque viverra.

Amet porttitor eget dolor morbi non. Adipiscing diam donec adipiscing tristique risus. In massa tempor nec feugiat nisl pretium fusce. Habitasse platea dictumst quisque sagittis purus sit. Morbi leo urna molestie at elementum eu facilisis sed. Volutpat sed cras ornare arcu dui. Sit amet purus gravida quis blandit turpis cursus in hac. Hendrerit gravida rutrum quisque non tellus orci ac \cite{Reno2016,Andresen2016}. Vehicula ipsum a arcu cursus vitae congue mauris. Facilisi etiam dignissim diam quis enim lobortis scelerisque fermentum dui. Facilisis gravida neque convallis a cras semper. Enim diam vulputate ut pharetra. Nulla pellentesque dignissim enim sit amet. Bibendum ut tristique et egestas quis.

\lipsum[45-47]

\section{Perumusan Masalah}
\lipsum[10]
\begin{enumerate}
    \item \lipsum[19]
    \item \lipsum[20]
    \item \lipsum[21]
\end{enumerate}

\section{Tujuan Penelitian}
\lipsum[23]
\begin{enumerate}
    \item \lipsum[25]
    \item \lipsum[27]
    \item \lipsum[29]
\end{enumerate}

\section{Manfaat Penelitian}
\lipsum[77]

\chapter{TINJAUAN PUSTAKA DAN LANDASAN TEORI}
\lipsum[90]
\section{Tinjauan Pustaka}
Nibh tortor id aliquet lectus. Euismod lacinia at quis risus. Morbi blandit cursus risus at ultrices mi tempus \cite{Mortazavi2015,Hasheminamin2015}. Risus viverra adipiscing at in tellus integer feugiat. Neque volutpat ac tincidunt vitae semper quis lectus nulla at \cite{Abdelkader2020}. Aenean pharetra magna ac placerat vestibulum lectus mauris ultrices eros. Orci porta non pulvinar neque laoreet suspendisse interdum consectetur libero. Pellentesque diam volutpat commodo sed egestas egestas. Ut enim blandit volutpat maecenas. Tellus molestie nunc non blandit massa enim nec dui. Nunc scelerisque viverra mauris in aliquam sem fringilla ut. At auctor urna nunc id. Ut placerat orci nulla pellentesque. Imperdiet proin fermentum leo vel orci porta non. Duis ut diam quam nulla porttitor massa. Justo donec enim diam vulputate ut pharetra sit amet aliquam. Turpis egestas pretium aenean pharetra magna \cite{Breker2015}.

At ultrices mi tempus imperdiet nulla malesuada pellentesque elit. Eget mi proin sed libero enim sed faucibus. Enim ut tellus elementum sagittis \cite{Beck2016}. Aenean vel elit scelerisque mauris pellentesque pulvinar pellentesque habitant morbi. Malesuada bibendum arcu vitae elementum curabitur vitae nunc. Aliquet enim tortor at auctor urna nunc id cursus metus. Morbi tristique senectus et netus. Semper eget duis at tellus at urna. Sed sed risus pretium quam vulputate. Dolor sit amet consectetur adipiscing elit pellentesque habitant. Urna id volutpat lacus laoreet non curabitur. Dui ut ornare lectus sit amet est placerat in \cite{Gooding2014,Breker2015,Wang2020}. Porttitor massa id neque aliquam. Augue eget arcu dictum varius duis. Porttitor eget dolor morbi non arcu risus quis varius quam. Pellentesque habitant morbi tristique senectus et netus et malesuada. Elementum nibh tellus molestie nunc non blandit massa enim nec. Morbi tristique senectus et netus et malesuada. Curabitur vitae nunc sed velit dignissim sodales ut eu \cite{Hasheminamin2015}. Nec nam aliquam sem et \cite{Atmaja2022}.
\subsection{Vestibulum Ante}
\lipsum[91] Gambar \ref{fig:ros}.
\begin{figure}[!h]
	\vspace*{0pt}
	\centering
	\includegraphics[width=0.7\textwidth]{fig/wiregrid}
	\caption{Eget eros lectus luctus.}
	\label{fig:ros}
\end{figure}
\subsubsection{Ipsum Primis}
\lipsum[81] Persamaan (\ref{eq:vol}).
\begin{equation}\label{eq:vol}
Q_{source} = \dfrac{(E_S - E_R)(E_S + E_R)}{R+jX}.
\end{equation}
\subsubsection{Etiam Massa}
\lipsum[27]
\subsection{Vitae Suspendisse Nisl}
\lipsum[91-92] 
\subsubsection{Morbi Malesuada}
\lipsum[80]
\subsubsection{Scing Etiam Lacus}
\lipsum[70] Tabel \ref{tab:maec}.
\bgroup
\vspace{4pt}
{\renewcommand{\arraystretch}{1.3}
\begin{table}[!h]
	\caption{Nunc Maecenas Augue Velit Aliquam Eros}
	\vspace{-12pt}
	\begin{center}
		\begin{tabular}{|@{\hspace*{0.7em}\extracolsep{\fill}}p{14em}@{\hspace*{0.7em}\extracolsep{\fill}}|@{\hspace*{0.7em}\extracolsep{\fill}}p{14em}@{\hspace*{0.7em}\extracolsep{\fill}}|}
			\cline{1-2} 
			\textbf{Nunc Maecenas}&	
			\textbf{Eros}\\
			\hline Temporibus autem quibusdam et aut officiis debitis aut rerum ut et voluptates repudiandae sint et molestiae non recusandae.  &Itaque earum rerum hic tenetur a sapiente delectus, ut aut reiciendis voluptatibus maiores alias consequatur aut perferendis doloribus asperiores repellat\\
			\hline Voluptatibus maiores alias consequatur aut perferendis doloribus asperiores repellat.  &Qui officia deserunt mollitia animi, id est laborum et dolorum fuga\\
			\hline
		\end{tabular}
		\label{tab:maec}
	\end{center}
	\vspace{-12pt}
\end{table}
\egroup
\section{Landasan Teori}
\lipsum[29]
\subsection{Amet Semper Nulla}
\lipsum[92]
\subsection{Laoreet Non}
\lipsum[89]

\section{Hipotesis}

Ante proin auctor duis vehicula etiam justo:
\begin{enumerate}
    \item \lipsum[34]
    \item \lipsum[35]
\end{enumerate} 

%!TEX root = /Users/wyatmaja/Documents/Doctor/Format LaTex/latexdisertasidtetiugm/disertasidtetiugm.tex
\chapter{METODE PENELITIAN}
\lipsum[67] Tabel \ref{tab:vel}.
\bgroup
\vspace{4pt}
{\renewcommand{\arraystretch}{1.3}
\begin{table}[!h]
	\caption{Proin Sagittis Sit Amet}
	\vspace{-12pt}
	\begin{center}
		\begin{tabular}{|@{\hspace*{0.7em}\extracolsep{\fill}}p{14em}@{\hspace*{0.7em}\extracolsep{\fill}}|@{\hspace*{0.7em}\extracolsep{\fill}}p{14em}@{\hspace*{0.7em}\extracolsep{\fill}}|}
			\cline{1-2} 
			\textbf{Vel Urna Rhoncus}&	
			\textbf{Tellus Quam}\\
			\hline Ullamcorper sit amet risus nullam.  &Magna etiam tempor lobortis.\\
			\hline A erat nam at lectus urna duis. &Eleifend quam adipiscing vitae proin sagittis nisl rhoncus mattis.\\
			\hline  Viverra adipiscing at in tellus integer feugiat. &Bibendum at varius vel pharetra.\\
			\hline
		\end{tabular}
		\label{tab:vel}
	\end{center}
	\vspace{-12pt}
\end{table}
\egroup
\section{Adipiscing Elit Elementum}
\lipsum[48-50] Persamaan (\ref{eq:von}).
\begin{equation}\label{eq:von}
	\left( \begin{array}{c|c|c|r}
	a+b+c & uv & x-y & 27\\ \hline
	a+b & u+v & z & 134
	\end{array}\right)
\end{equation}
\subsection{Orci Tristique Eu}
\lipsum[44-47]
\subsubsection{Malesuada Donec Ulrices Dui}
\lipsum[56-58] Gambar \ref{fig:nam}.
\begin{figure}[!h]
	\vspace*{0pt}
	\centering
	\includegraphics[width=0.9\textwidth]{fig/smartgrid}
	\caption{Etiam vital nam in.}
	\label{fig:nam}
\end{figure}

\subsubsection{Bibendum Consectetuer Vivamus}
\lipsum[34-36]
%!TEX root = /Users/wyatmaja/Documents/Doctor/proposalhc/dissertationproposal.tex
\chapter{KEMAJUAN PENELITIAN}
Bab ini menyediakan kemajuan-kemajuan penelitian yang sudah didapatkan. Penelitian ini dimulai dengan merancang metode penentuan kapasitas \textit{hosting} stokastik untuk penetrasi PLTS tersebar dengan menggunakan Markov Chain Monte Carlo sederhana, belum menggunakan \textit{hidden Markov model}.

\section{Rancangan Skenario Penetrasi PLTS Tersebar}

%\section{Pengkajian Pustaka dan Persiapan Data Masukan}
%\section{Formulasi Permasalahan, Perancangan Metode, dan Pengerjaan Simulasi}
%\subsection{Kapasitas \textit{Hosting} Stokastik}
%\subsection{Batas Kinerja Operasi}
%\subsection{Kerangka Analisis Runtun Waktu}
%\subsection{Perancangan Metode Penentuan Kapasitas \textit{Hosting} Stokastik}
%\subsubsection{Perancangan Skenario Penetrasi PLTS Tersebar}
\begin{enumerate}
    \item \textit{Skenario penetrasi}:
    Skenario penetrasi $A^m$ adalah skenario ke-$m$ dalam mempenetrasikan PLTS ke jaringan distribusi tenaga listrik. Skenario penetrasi $A^m$ dirumuskan dalam himpunan $A$, yaitu
    \begin{equation}\label{pas1}
    %A=\{A^m\ |\ m\in M\},
    A=\{A^1,A^2,...,A^m,...,A^u\}.
    \end{equation}
    
    Dalam penelitian ini, jumlah semua skenario penetrasi $u$ juga merepresentasikan jumlah simulasi Markov Chain Monte Carlo.
    
    \item \textit{States dari Markov Chain Monte Carlo}:
    Pada penelitian ini, \textit{states} dari Markov Chain Monte Carlo merepresentasikan kluster yang mungkin dari pelanggan, yaitu
    \begin{equation}\label{smc1}
    Y=\{y|y \text{ adalah kluster dari pelanggan}\}.
    \end{equation} 
    
    Dalam hal ini, \textit{state} $y_i$ adalah \textit{state} sebelumnya dari $y_j$. Keduanya merupakan nilai keluaran dari $y$.
    
    \item \textit{Proses penetrasi}:
    Skenario penetrasi ke-$m$ terdiri atas sejumlah PLTS dengan ukuran daya dan lokasi penetrasi yang berbeda-beda. Untuk setiap skenario penetrasi, proses melakukan penetrasi ke-$\{j,k\}$ yang merepresentasikan ukuran daya dan lokasi penetrasi disebut sebagai proses penetrasi $a^m_{j,k}$. Hal ini dirumuskan sebagai
    \begin{equation}\label{pas2}
    %A^m=\{a^m_{j,k}\ |\ m\in M\wedge k\in K\}.
    A^m=\{a^m_{j,1},a^m_{j,2},...,a^m_{j,k},...,a^m_{j,w}\}.
    \end{equation} 
    
    Dengan skenario penetrasi sebanyak $u$ dan proses penetrasi sebanyak $w$ maka dihasilkan total kombinasi acak dari ukuran daya dan lokasi penetrasi sebanyak $u\bigcdot w$.
    
    \item \textit{Pelanggan penetrasi}:
    Dalam penelitian ini, pelanggan penetrasi $c^m_{j,k}$ didefinisikan sebagai pelanggan saat ini yang dipilih secara acak untuk mengintegrasikan PLTS ke jaringan distribusi tenaga listrik. Pelanggan penetrasi dipilih untuk proses penetrasi ke-$\{m,j,k\}$. Hal ini didefinisikan sebagai
    \begin{equation}\label{cpv1}
    %\mathit{C^m_j}=\{c^m_{j,k}\ |\ m\in M\wedge k\in K\wedge j\in \Z^+\}.
    \mathit{C^m_j}=\{c^m_{j,1},c^m_{j,2},...,c^m_{j,k},...,c^m_{j,w}\}.
    \end{equation}
    
    Untuk setiap proses penetrasi ke-${k}$, dipilih satu pelanggan penetrasi $c^m_{j,k}$. Oleh karena itu, jika simulasi Markov Chain Monte Carlo sebanyak $u$ dikerjakan maka terdapat skenario penetrasi sebanyak $u$ untuk setiap pelanggan penetrasi. Hal ini dirumuskan sebagai
    \begin{equation}\label{cpv2}
    \mathit{C^k_j}=\{c^k_{j,1},c^k_{j,2},...,c^k_{j,m},...,c^k_{j,u}\}.
    \end{equation}
    
    \item \textit{Ukuran daya penetrasi}:
    Ukuran daya penetrasi PLTS $x^m_{j,k}$ ditentukan secara bersamaan. Untuk pelanggan yang hanya mengintegrasikan PLTS, ukuran daya yang digunakan hanya PLTS. Pada pelanggan yang mengintegrasikan PLTS dan baterai, baik PLTS maupun baterai ditentukan ukuran daya masing-masing. Ukuran daya penetrasi $x^m_{j,k}$ didasarkan pada skenario ke-$m$, kluster ke-$j$, dan proses penetrasi ke-$k$. Dengan kata lain, ukuran daya penetrasi $x^m_{j,k}$ adalah ukuran daya dari PLTS untuk pelanggan penetrasi $c^m_{j,k}$. Ukuran daya penetrasi dipilih dari data historis $D$ melalui proses Markov Chain Monte Carlo. Ukuran daya penetrasi $x^m_{j,k}$ dirumuskan dalam himpunan ukuran daya penetrasi $X^m_{j}$ sebagai
    %\begin{equation}\label{xpv1}
    %x^m_{j,k}=\{x^m_{j,k}\ \text{is random value}|y_{j-1} < x^m_{j,k} %\leq y_j\},
    %\end{equation}
    \begin{equation}\label{xpv2}
    %\mathit{X^m_j}=\{x^m_{j,k}\ |\ y_{j-1} < x^m_{j,k} \leq y_j\wedge x^m_{j,k}\in D\wedge m\in M\wedge k\in K\wedge y_j\in Y\wedge j\in \Z^+\}.
    \mathit{X^m_j}=\{x^m_{j,1},x^m_{j,2},...,x^m_{j,k},...,x^m_{j,w}\},
    \end{equation}
    dan dapat dirumuskan sebagai
    \begin{equation}\label{xpv3}
    \mathit{X^k_j}=\{x^k_{j,1},x^k_{j,2},...,x^k_{j,m},...,x^k_{j,u}\}.
    \end{equation}
    
    \item \textit{Level penetrasi pelanggan}:
    Untuk setiap skenario penetrasi, level penetrasi pelanggan $cp^m_k$ adalah jumlah total dari pelanggan-pelanggan penetrasi dari awal simulasi sampai proses penetrasi ke-$k$. Secara matematis, level penetrasi pelanggan $cp^m_k$ didefinisikan sebagai
    \begin{equation}\label{cpl1}
    cp^m_k=k,
    \end{equation}
    dan untuk semua penetrasi,
    \begin{equation}\label{cpl2}
    cp^m_w=w,
    \end{equation}
    kemudian, himpunan level penetrasi pelanggan $\mathit{CP^m}$ dirumuskan sebagai
    \begin{equation}\label{cpl3}
    %\mathit{CP^m}=\{cp^m_k\ |\ m\in M\wedge k\in K\}.
    \mathit{CP^m}=\{cp^m_1,cp^m_2,...,cp^m_k,...,cp^m_w\}.
    \end{equation}
    \item \textit{Level penetrasi daya}:
    Jumlah total ukuran daya yang telah dipenetrasikan ke jaringan distribusi tenaga listrik ke-$\{m,k\}$ didefinisikan sebagai level penetrasi daya $xp^m_k$. Level penetrasi daya $xp^m_k$ didefinisikan sebagai
    \begin{equation}\label{xpl1}
    xp^m_k=\sum_{k=1}^k x^m_{j,k},
    \end{equation}
    jika telah tercapai 100\% penetrasi maka
    \begin{equation}\label{xpl2}
    \mathit{xp^m_w}=\sum_{k=1}^w x^m_{j,k},
    \end{equation}
    kemudian, himpunan level penetrasi daya $\mathit{XP^m}$ dirumuskan sebagai
    \begin{equation}\label{xpl3}
    %\mathit{XP^m}=\{xp^m_k\ |\ m\in M\wedge k\in K\}.
    \mathit{XP^m}=\{xp^m_1,xp^m_2,...,xp^m_k,...,xp^m_w\}.
    \end{equation}
    
    \item \textit{Lokasi penetrasi}:
    Lokasi penetrasi $l^m_{j,k}$  didasarkan pada lokasi pelanggan penetrasi $c^m_{j,k}$ yang dipilih. Oleh karena itu, lokasi penetrasi $l^m_{j,k}$ dirumuskan sebagai
    \begin{equation}\label{lpv1}
    l^m_{j,k}=location(c^m_{j,k}),
    \end{equation}
    dan himpunan lokasi penetrasi $l^m_{j,k}$ dapat didefinisikan sebagai
    \begin{equation}\label{lpv2}
    %\mathit{L^m_j}=\{l^m_{j,k}\ |\ m\in M\wedge k\in K\wedge j\in \Z^+\}.
    \mathit{L^m_j}=\{l^m_{j,1},l^m_{j,2},...,l^m_{j,k},...,l^m_{j,w}\}.
    \end{equation}
    \item \textit{Daya keluaran PLTS}: Himpunan daya-daya keluaran PLTS untuk suatu rentang waktu pada skenario penetrasi ke-$m$ dan proses penetrasi ke-$k$ didefinisikan sebagai daya keluaran PLTS $e^m_{k}$. Untuk skenario penetrasi ke-$m$, daya-daya keluaran PLTS diformulasikan sebagai
    \begin{equation}\label{dkp1}
        \mathit{E^m}=\{E^m_{1},E^m_{2},...,E^m_{k},...,E^m_{w}\}.
    \end{equation}
    dengan daya keluaran PLTS $E^m_{k}$ dideskripsikan sebagai
    \begin{equation}\label{dkp2}
        \mathit{E^m_k}=\{e^m_{k,1},e^m_{k,2},...,e^m_{k,t},...,e^m_{w,v}\}.
    \end{equation}
    \item \textit{Permintaan beban}: Permintaan beban $g^m_{k}$ dideskripsikan sebagai himpunan permintaan-permintaan beban untuk suatu rentang waktu pada skenario penetrasi ke-$m$ dan proses penetrasi ke-$k$. Untuk skenario penetrasi ke-$m$, permintaan-permintaan beban dirumuskan sebagai
    \begin{equation}\label{pb1}
        \mathit{G^m}=\{G^m_{1},G^m_{2},...,G^m_{k},...,G^m_{w}\}.
    \end{equation}
    dengan permintaan beban $G^m_{k}$ diformulasikan sebagai
    \begin{equation}\label{pb2}
        \mathit{G^m_k}=\{g^m_{k,1},g^m_{k,2},...,g^m_{k,t},...,g^m_{w,v}\}.
    \end{equation}
    
    \end{enumerate}
    
\section{Algoritme}
Dengan menggunakan diagram alir penentuan kapasitas \textit{hosting} stokastik untuk PLTS tersebar yang ditunjukkan pada Gambar \ref{fcmcmc}, dan dengan skenario-skenario penetrasi PLTS yang telah dirancang, Algoritme \ref{alg:marchai1} dirancang.

%\subsubsection{Perancangan Markov Chain Monte Carlo}
%\subsubsection{Pembentukan Algoritme}
\bgroup
\begin{algorithm}[htbp]
	\caption{Markov Chain Monte Carlo untuk Penentuan Kapasitas \textit{Hosting} Stokastik}\label{alg:marchai1}
	\begin{algorithmic}[1]
		\Procedure{MakeSetPhotovoltaicPenetration}{}
		\State Create sorted data set $\mathit{D}$
		\EndProcedure \\
		
		\Procedure{MakeSetMarkovStates}{}
		\State\textproc {MakeSetPhotovoltaicPenetration}
		\State Create set $Y$ from set $D$
		\EndProcedure \\
		
		\Procedure{ObtainLoadCurve}{}
		\State Obtain load curve containing load demands in time-series
		\EndProcedure \\
		
		\Procedure{ObtainPhotovoltaicOutput}{}
		\State Obtain photovoltaic output in time-series
		\EndProcedure \\	
		
		\Procedure{MakeSetPhotovoltaicAllocation}{}
		\State Create set $A$ using Eq. (\ref{pas1})
		\State Create set $A^m$ using Eq. (\ref{pas2})
		\EndProcedure \\
		
		\Procedure {MakeSetCustomerWithPhotovoltaic}{}
		\State Create $\mathit{C^m_j}$ using Eq. (\ref{cpv1})
		\EndProcedure\\
		
		\Procedure{MakeSetPhotovoltaicSize}{}
		\State Create $\mathit{X^m_j}$ using Eq. (\ref{xpv2})
		\EndProcedure\\

		\Function{CalculateCP}{}
		\State Calculate $cp^m_k$ using Eq. (\ref{cpl1})
		\State\Return $cp^m_k$
		\EndFunction \\
		
		\Function{CalculateXP}{}
		\State Calculate $xp^m_k$ using Eq. (\ref{xpl1})
		\State\Return $xp^m_k$
		\EndFunction \\
		
		\Procedure{MakeSetIntegrationLocation}{}
		\State  Create $\mathit{L^m_j}$ using Eq. (\ref{lpv2})
		\EndProcedure

		\algstore{bkbreak}
	\end{algorithmic}
\end{algorithm}
\addtocounter{algorithm}{-1}
\begin{algorithm}[htbp]
	\caption{Markov Chain Monte Carlo untuk Penentuan Kapasitas \textit{Hosting} Stokastik}
	\begin{algorithmic}[1]
	\algrestore{bkbreak}
		\Function{CalculateH}{}
		\State Calculate $\mathit{h^m}$ using Eq. (\ref{phc1})
		\State\Return $\mathit{h^m}$
		\EndFunction\\

		\Function {TransProbMatrix}{}
		\State\textproc {MakeSetMarkovStates}		
		\For {each $\{y_i,y_j\}$ in $Y$}
		\State $y_i \gets$ prior outcome value
		\State $y_j \gets$ posterior outcome value
		\State Calculate $p_{ij}$ using Eq. (\ref{tp2})
		\EndFor 		
		\State Create $r\times r$ transition probability matrix $\mathbf{P}$ with $p_{ij}$ as entries
		\State\Return matrix $\mathbf{P}$
		\EndFunction\\

		\Function{PerformPhotovoltaicAllocation}{}
		\If {$r$ changes}
		\State Update $Y$ with remaining $y$
		\EndIf
		\For {each $k$ in rooftop photovoltaic process $a^m_k$}
		\State\textproc {TransProbMatrix}($a^m_k$)
		\State Calculate state $y_j$ given state $y_i$ using matrix $\mathbf{P}$
		\EndFor
		\State\Return State $y_j$
		\EndFunction\\
		
		\Function{PerformMarkovChainsMonteCarlo}{}
		\State\textproc{ObtainLoadCurve}
		\State\textproc{ObtainPhotovoltaicOutput}
		\State Construct a code of the distribution grid under study
		\State Define the operational limits concerned
		\State Initialize  $m\ :\ m=1$
		\State Initialize  $k\ :\ k=1$
		\State Obtain $a^m_k$
		\State\textproc {MakeSetPhotovoltaicAllocation}
		\State Initialize $y_j$
		\State Obtain $c^m_{j,k}$, $x^m_{j,k}$, and $l^m_{j,k}$
		\State\textproc {MakeSetCustomerWithPhotovoltaic}
		\State\textproc {MakeSetPhotovoltaicSize}
		\State\textproc {MakeSetIntegrationLocation}
		\While{$m$ in $cp^m_k \neq$ $u$}
		\State\textproc{CalculateCP}
		\State\textproc{CalculateXP}
		\algstore{bkbreak}
	\end{algorithmic}
\end{algorithm}	
\addtocounter{algorithm}{-1}
\begin{algorithm}[!t]
	\caption{Markov Chain Monte Carlo untuk Penentuan Kapasitas \textit{Hosting} Stokastik}
	\begin{algorithmic}[1]
	\algrestore{bkbreak}
		\If{$m$ $>=$ 2}
		\State Update $a^m_k$
		\EndIf	

		\While {$k$ in $cp^m_k \neq$ $w$}		
		\If{$k$ $>=$ 2}	
		\State $y_i \gets$ $y_j$

		\State\textproc {PerformPhotovoltaicAllocation}
		\State Determine $c^m_{j,k}$, $x^m_{j,k}$, and $l^m_{j,k}$ 
		\EndIf	
		\State Penetrate rooftop photovoltaic using $c^m_{j,k}$, $x^m_{j,k}$, and $l^m_{j,k}$
		\State Run power flow in time-series
		\State Evaluate the results using the operational limits
		\If {violation occurs at $\mathit{xp^m_k}$}
		\State Determine $\mathit{xp^m_{k-\text{1}}}$ as rooftop photovoltaic hosting capacity
		\EndIf		
		\State Update $k\ :\ k=k+1$
		\EndWhile
		\State Calculate $\mathit{h_{min}}$ using Eq. (\ref{phc2})
		\State Update $m\ :\ m=m+1$
		\EndWhile		
		\State\Return $\mathit{A^m}$, $\mathit{C^m_j}$, $\mathit{X^m_j}$, $\mathit{CP^m}$, $\mathit{XP^m}$, $\mathit{L^m_j}$, and $\mathit{h_{min}}$
		\State Record the desired results
		\EndFunction
	\end{algorithmic}
\end{algorithm}
\egroup

%\subsection{Pengembangan Metode dengan Pertimbangan PLTS-Baterai Tersebar}
%\section{Penganalisisan Hasil dan Penarikan Kesimpulan}
%!TEX root = /Users/wyatmaja/Documents/Doctor/proposalhc/dissertationproposal.tex
\chapter{JADWAL PENELITIAN}
%\bgroup
%\renewcommand{\arraystretch}{1.3}
%\begin{sidewaystable}[!h]
%    \vspace{-14pt}
%	\begin{center}
%        \begin{tabular}{|@{\hspace*{0.2em}\extracolsep{\fill}}p{15em}@{\hspace*{0.2em}\extracolsep{\fill}} | l | l | l | l | l | l | l | l | l | l | l | l | l | l | l | l | l | l |}
%        \hline
%        \multirow{2}{*}{\textbf{Aktivitas}} & \multicolumn{6}{c|}{\textbf{2020 (Semester 1)}} & \multicolumn{6}{c|}{\textbf{2021 (Semester 2)}} & \multicolumn{6}{c|}{\textbf{2021 (Semester 3)}}\\
%        \cline{2-19} 
%        &
%		\textbf{7 }&
%		\textbf{8 }&
%        \textbf{9 }&
%		\textbf{10}&
%        \textbf{11}&
%		\textbf{12}&
%        \textbf{1 }&
%		\textbf{2 }&
%        \textbf{3 }&
%		\textbf{4 }&
%        \textbf{5 }&
%		\textbf{6 }&
%        \textbf{7 }&
%		\textbf{8 }&
%        \textbf{9 }&
%		\textbf{10}&
%        \textbf{11}&
%		\textbf{12}\\\hline
%        Studi pustaka dan persiapan & \cellcolor{bleudefrance} & \cellcolor{bleudefrance} & \cellcolor{bleudefrance} & \cellcolor{bleudefrance} & & & & & & & & & & & & & & \\\hline 
%        Perumusan masalah dan hipotesis & \cellcolor{bleudefrance} & \cellcolor{bleudefrance} & \cellcolor{bleudefrance} & \cellcolor{bleudefrance} & & & & & & & & & & & & & & \\\hline
%        Pengambilan data & & & \cellcolor{bleudefrance} & \cellcolor{bleudefrance} & \cellcolor{bleudefrance} & & & & & & & & & & & & & \\\hline
%        Pemodelan sistem distribusi & & & \cellcolor{bleudefrance} & \cellcolor{bleudefrance} & \cellcolor{bleudefrance} & \cellcolor{bleudefrance} & & & & & & & & & & & & \\\hline
%        Uji validitas model sistem distribusi & & & & \cellcolor{bleudefrance} & \cellcolor{bleudefrance} & \cellcolor{bleudefrance} & & & & & & & & & & & & \\\hline
%        Penyusunan proposal & & & & & \cellcolor{bleudefrance} & \cellcolor{bleudefrance} & \cellcolor{bleudefrance} & \cellcolor{bleudefrance} & & & & & & & & & & \\\hline
%        Perancangan model kapasitas \textit{hosting} dengan Markov Chain Monte Carlo & & & & & & & \cellcolor{bleudefrance} & \cellcolor{bleudefrance} & \cellcolor{bleudefrance} & & & & & & & & & \\\hline
%        Pengujian, evaluasi, dan perbaikan & & & & & & & & \cellcolor{bleudefrance} & \cellcolor{bleudefrance} & \cellcolor{bleudefrance} & & & & & & & & \\\hline
%        Ujian komprehensif & & & & & & & & & \cellcolor{bleudefrance} & \cellcolor{bleudefrance} & & & & & & & & \\\hline
%        Analisis dan pembahasan & & & & & & & & & \cellcolor{bleudefrance} & \cellcolor{bleudefrance} & \cellcolor{bleudefrance} & & & & & & & \\\hline
%        Penyusunan dan pengiriman artikel I & & & & & & & & & & \cellcolor{bleudefrance} & \cellcolor{bleudefrance} & \cellcolor{bleudefrance} & & & & & & \\\hline
%
%        Perancangan model kapasitas \textit{hosting} dengan hidden Markov model & & & & & & & & & & & & & \cellcolor{bleudefrance} & \cellcolor{bleudefrance} & \cellcolor{bleudefrance} & & & \\\hline
%        Pengujian, evaluasi, dan perbaikan & & & & & & & & & & & & & & \cellcolor{bleudefrance} & \cellcolor{bleudefrance} & \cellcolor{bleudefrance} & & \\\hline
%        Analisis dan pembahasan & & & & & & & & & & & & & & & \cellcolor{bleudefrance} & \cellcolor{bleudefrance} & \cellcolor{bleudefrance} & \\\hline
%    \end{tabular}
%    \end{center}
%    \vspace{-16pt}
%\end{sidewaystable}
%\egroup
%
%\bgroup
%\renewcommand{\arraystretch}{1.3}
%\begin{sidewaystable}[!h]
%    \vspace{-14pt}
%	\begin{center}
%        \begin{tabular}{|@{\hspace*{0.2em}\extracolsep{\fill}}p{15em}@{\hspace*{0.2em}\extracolsep{\fill}} | l | l | l | l | l | l | l | l | l | l | l | l | l | l | l | l | l | l |}
%        \hline
%        \multirow{2}{*}{\textbf{Aktivitas}} & \multicolumn{6}{c|}{\textbf{2022 (Semester 4)}} & \multicolumn{6}{c|}{\textbf{2022 (Semester 5)}} & \multicolumn{6}{c|}{\textbf{2023 (Semester 6)}}\\
%        \cline{2-19} 
%        &
%		\textbf{1 }&
%		\textbf{2 }&
%        \textbf{3 }&
%		\textbf{4 }&
%        \textbf{5 }&
%		\textbf{6 }&
%        \textbf{7 }&
%		\textbf{8 }&
%        \textbf{9 }&
%		\textbf{10}&
%        \textbf{11}&
%		\textbf{12}&
%        \textbf{1 }&
%		\textbf{2 }&
%        \textbf{3 }&
%		\textbf{4 }&
%        \textbf{5 }&
%		\textbf{6 }\\\hline
%        Penyusunan dan pengiriman artikel II & \cellcolor{bleudefrance} & \cellcolor{bleudefrance} & \cellcolor{bleudefrance} & & & & & & & & & & & & & & & \\\hline
%        Seminar hasil I & & & \cellcolor{bleudefrance} & \cellcolor{bleudefrance} & & & & & & & & & & & & & & \\\hline
%        Pengembangan model kapasitas \textit{hosting} dengan PLTS-baterai & & & & \cellcolor{bleudefrance} & \cellcolor{bleudefrance} & \cellcolor{bleudefrance} & & & & & & & & & & & & \\\hline
%        Pengujian, evaluasi, dan perbaikan & & & & & \cellcolor{bleudefrance} & \cellcolor{bleudefrance} & \cellcolor{bleudefrance} & & & & & & & & & & & \\\hline
%        Analisis dan pembahasan & & & & & & \cellcolor{bleudefrance} & \cellcolor{bleudefrance} & \cellcolor{bleudefrance} & & & & & & & & & & \\\hline
%        Penyusunan dan pengiriman artikel III & & & & & & & \cellcolor{bleudefrance} & \cellcolor{bleudefrance} & \cellcolor{bleudefrance} & & & & & & & & & \\\hline
%
%        Seminar hasil II & & & & & & & & \cellcolor{bleudefrance} & \cellcolor{bleudefrance} & & & & & & & & & \\\hline
%        Pengembangan metode final & & & & & & & & & \cellcolor{bleudefrance} & \cellcolor{bleudefrance} & \cellcolor{bleudefrance} & & & & & & & \\\hline
%        Pengujian, evaluasi, dan perbaikan & & & & & & & & & & \cellcolor{bleudefrance} & \cellcolor{bleudefrance} & \cellcolor{bleudefrance} & & & & & & \\\hline
%        Analisis dan pembahasan & & & & & & & & & & & \cellcolor{bleudefrance} & \cellcolor{bleudefrance} & \cellcolor{bleudefrance} & & & & & \\\hline
%        Penyusunan dan pengiriman artikel IV & & & & & & & & & & & & \cellcolor{bleudefrance} & \cellcolor{bleudefrance} & \cellcolor{bleudefrance} & & & & \\\hline
%        Perbaikan dan penyelesaian laporan & & & & & & & & & & & & & & \cellcolor{bleudefrance} & \cellcolor{bleudefrance} & \cellcolor{bleudefrance} & \cellcolor{bleudefrance} & \\\hline
%        Ujian tertutup & & & & & & & & & & & & & & & & & \cellcolor{bleudefrance} & \cellcolor{bleudefrance} \\\hline
%    \end{tabular}
%    \end{center}
%    \vspace{-16pt}
%\end{sidewaystable}
%\egroup

\bgroup
\vspace{4pt}
%\def\arraystretch{1.3}
\renewcommand{\arraystretch}{1.1}
\begin{table}[!h]
    \vspace{-14pt}
	\begin{center}
        \begin{tabular}{|@{\hspace*{0.2em}\extracolsep{\fill}}p{12em}@{\hspace*{0.2em}\extracolsep{\fill}} | l | l | l | l | l | l | l | l | l | l | l | l |}
        \hline
        \multirow{2}{*}{\textbf{Aktivitas}} & \multicolumn{6}{c|}{\textbf{2020 (Semester 1)}} & \multicolumn{6}{c|}{\textbf{2021 (Semester 2)}}\\
        \cline{2-13} 
		&
		\textbf{7 }&
		\textbf{8 }&
        \textbf{9 }&
		\textbf{10}&
        \textbf{11}&
		\textbf{12}&
        \textbf{1 }&
		\textbf{2 }&
        \textbf{3 }&
		\textbf{4 }&
        \textbf{5 }&
		\textbf{6 }\\\hline
        Studi pustaka dan persiapan & \cellcolor{bleudefrance} & \cellcolor{bleudefrance} & \cellcolor{bleudefrance} & & & & & & & & & \\\hline 
        Perumusan masalah dan hipotesis & & \cellcolor{bleudefrance} & \cellcolor{bleudefrance} & \cellcolor{bleudefrance} & & & & & & & & \\\hline
        Pengumpulan data & & & & \cellcolor{bleudefrance} & \cellcolor{bleudefrance} & & & & & & & \\\hline
        Pemodelan sistem distribusi & & & & \cellcolor{bleudefrance} & \cellcolor{bleudefrance} & \cellcolor{bleudefrance} & & & & & & \\\hline
        Uji validitas model sistem distribusi & & & & & \cellcolor{bleudefrance} & \cellcolor{bleudefrance} & & & & & & \\\hline
        \textbf{Tahap I:} Perancangan me- tode dengan \textit{Markov model} & & & & & & \cellcolor{bleudefrance} & \cellcolor{bleudefrance} & \cellcolor{bleudefrance} & \cellcolor{bleudefrance} & & & \\\hline
        Simulasi, evaluasi, dan perbaikan & & & & & & & \cellcolor{bleudefrance} & \cellcolor{bleudefrance} & \cellcolor{bleudefrance} & & & \\\hline
        Ujian komprehensif & & & & & & & & & \cellcolor{bleudefrance} & \cellcolor{bleudefrance} & & \\\hline
        %Analisis dan pembahasan & & & & & & & & & \cellcolor{bleudefrance} & \cellcolor{bleudefrance} & \cellcolor{bleudefrance} & \\\hline
        Penyusunan dan pengiriman artikel I & & & & & & & & & & \cellcolor{bleudefrance} & \cellcolor{bleudefrance} & \cellcolor{bleudefrance} \\\hline
    \end{tabular}
    \end{center}
    \vspace{-16pt}
\end{table}
\egroup
\bgroup
\vspace{4pt}
%\def\arraystretch{1.3}
\renewcommand{\arraystretch}{1.1}
\begin{table}[!h]
    \vspace{-14pt}
	\begin{center}
        \begin{tabular}{|@{\hspace*{0.2em}\extracolsep{\fill}}p{12em}@{\hspace*{0.2em}\extracolsep{\fill}} | l | l | l | l | l | l | l | l | l | l | l | l |}
        \hline
        \multirow{2}{*}{\textbf{Aktivitas}} & \multicolumn{6}{c|}{\textbf{2021 (Semester 3)}} & \multicolumn{6}{c|}{\textbf{2022 (Semester 4)}}\\
        \cline{2-13} 
        &
		\textbf{7 }&
		\textbf{8 }&
        \textbf{9 }&
		\textbf{10}&
        \textbf{11}&
		\textbf{12}&
        \textbf{1 }&
		\textbf{2 }&
        \textbf{3 }&
		\textbf{4 }&
        \textbf{5 }&
		\textbf{6 }\\\hline
        \textbf{Tahap II:} Pengembangan metode dengan \textit{hidden Markov model}& \cellcolor{bleudefrance} & \cellcolor{bleudefrance} & \cellcolor{bleudefrance} & \cellcolor{bleudefrance} & & & & & & & & \\\hline
        Simulasi, evaluasi, dan perbaikan & & \cellcolor{bleudefrance} & \cellcolor{bleudefrance} & \cellcolor{bleudefrance} & & & & & & & & \\\hline
        %Analisis dan pembahasan & & & \cellcolor{bleudefrance} & \cellcolor{bleudefrance} & \cellcolor{bleudefrance} & & & & & & & \\\hline
        Penyusunan dan pengiriman artikel II & & & & \cellcolor{bleudefrance} & \cellcolor{bleudefrance} & \cellcolor{bleudefrance} & & & & & & \\\hline
        Seminar hasil I & & & & & & \cellcolor{bleudefrance} & \cellcolor{bleudefrance} & & & & & \\\hline
        \textbf{Tahap III:} Pengembangan metode dengan PLTS-baterai menggunakan \textit{layered hidden Markov model} & & & & & & & \cellcolor{bleudefrance} & \cellcolor{bleudefrance} & \cellcolor{bleudefrance} & \cellcolor{bleudefrance} & & \\\hline
        Simulasi, evaluasi, dan perbaikan & & & & & & & & \cellcolor{bleudefrance} & \cellcolor{bleudefrance} & \cellcolor{bleudefrance} & & \\\hline
        %Analisis dan pembahasan & & & & & & & & & \cellcolor{bleudefrance} & \cellcolor{bleudefrance} & \cellcolor{bleudefrance} & \\\hline
        Penyusunan dan pengiriman artikel III & & & & & & & & & & \cellcolor{bleudefrance} & \cellcolor{bleudefrance} & \cellcolor{bleudefrance} \\\hline
    \end{tabular}
    \end{center}
    \vspace{-16pt}
\end{table}
\egroup
\bgroup
\vspace{4pt}
%\def\arraystretch{1.3}
\renewcommand{\arraystretch}{1.1}
\begin{table}[!h]
    \vspace{-14pt}
	\begin{center}
        \begin{tabular}{|@{\hspace*{0.2em}\extracolsep{\fill}}p{12em}@{\hspace*{0.2em}\extracolsep{\fill}} | l | l | l | l | l | l | l | l | l | l | l | l |}
        \hline
        \multirow{2}{*}{\textbf{Aktivitas}} & \multicolumn{6}{c|}{\textbf{2022 (Semester 5)}} & \multicolumn{6}{c|}{\textbf{2023 (Semester 6)}}\\
        \cline{2-13} 
        &
		\textbf{7 }&
		\textbf{8 }&
        \textbf{9 }&
		\textbf{10}&
        \textbf{11}&
		\textbf{12}&
        \textbf{1 }&
		\textbf{2 }&
        \textbf{3 }&
		\textbf{4 }&
        \textbf{5 }&
		\textbf{6 }\\\hline
        Seminar hasil II & \cellcolor{bleudefrance} & \cellcolor{bleudefrance} & & & & & & & & & & \\\hline
        \textbf{Tahap IV:} Studi sensitivitas dari variabel penetrasi dan sistem distribusi terhadap kapasitas \textit{hosting} stokastik & & \cellcolor{bleudefrance} & \cellcolor{bleudefrance} & \cellcolor{bleudefrance} & \cellcolor{bleudefrance} & & & & & & & \\\hline
        Simulasi, evaluasi dan perbaikan & & & \cellcolor{bleudefrance} & \cellcolor{bleudefrance} & \cellcolor{bleudefrance} & & & & & & & \\\hline
        %Perbaikan metode & & & & \cellcolor{bleudefrance} & \cellcolor{bleudefrance} & \cellcolor{bleudefrance} & & & & & & \\\hline
        %Analisis dan pembahasan final & & & & & & \cellcolor{bleudefrance} & \cellcolor{bleudefrance} & \cellcolor{bleudefrance} & & & & \\\hline
        Penyusunan dan pengiriman artikel IV & & & & & \cellcolor{bleudefrance} & \cellcolor{bleudefrance} & \cellcolor{bleudefrance} & & & & & \\\hline
        Penyusunan dan penyelesaian laporan & & & & & & & \cellcolor{bleudefrance} & \cellcolor{bleudefrance} & \cellcolor{bleudefrance} & \cellcolor{bleudefrance} & & \\\hline
        Persiapan dan pelaksanaan ujian & & & & & & & & & & \cellcolor{bleudefrance} & \cellcolor{bleudefrance} & \cellcolor{bleudefrance} \\\hline
    \end{tabular}
    \end{center}
    \vspace{-16pt}
\end{table}
\egroup
\textcolor{white}{.}

\textcolor{white}{.}

%\include{contents/chapter-6/chapter-6}
%\include{contents/chapter-7/chapter-7}

%%%%=============================================================
%%%% BACK MATTER (BAGIAN AKHIR)
%%%%=============================================================

%%%%--------------------------------------
%%%% References
%%%%--------------------------------------
\thereferences
%%%% Export bib file from Mendeley/Zotero, and put it in the following brackets
\bibliography{doktordtetiugm}

%%%%--------------------------------------
%%%% Appendix
%%%%--------------------------------------
\chapterappendixone{contents/appendixone/appendix}
\chapterappendixtwo{contents/appendixtwo/appendix}
\newpage
\let\clearpage\relax
\end{document}